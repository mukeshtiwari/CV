%basic cover letter template
\documentclass{letter}
\usepackage{amssymb,amsmath}
\usepackage{graphicx}

\oddsidemargin=.2in
\evensidemargin=.2in
\textwidth=5.9in
\topmargin=-.5in
\textheight=9in

%\address{Mathematics Department\\University of Illinois\\
%1409 W. Green St\\Urbana, Illinois 61801}

\newcommand {\qed}{\mbox{$\Box$}}
\renewcommand {\iff}{\Longleftrightarrow}
\newcommand {\R}{\mathbb{R}}
\newcommand {\N}{\mathbb{N}}
\newcommand {\Q}{\mathbb{Q}}
\newcommand {\Z}{\mathbb{Z}}

\newcommand {\sub}{\mbox{SB}}


\begin{document}
\begin{letter}{Search Committee\\
CSIRO, Australia}


\opening{Dear Search Committee,}


I am writing to apply for the Research Scientist - Data Privacy and Confidentiality (67626). 
I am currently working as a Research Fellow with Toby Murray in the School of Computing and Information Systems, University 
of Melbourne, Melbourne. My research area is primarily focussed on formal verification of software program  which includes 
electronic voting and cryptography. 

As a Research Fellow at the University of Melbourne, I am investigating the 
information flow security in weak memory model in the presence of concurrency with Toby Murray. 
Verifying the correctness of functional programs without any side effect in a theorem prover 
is not very difficult. Moreover, in most of cases, the proofs are simple pen-and-paper proof. 
However, in case of side effects, the reasoning about the correctness of program is
difficult, mainly in the context of heap manipulating programs, and this situation gets 
more complicated in the presence of concurrency in weak-memory. 
Our project is geared 
towards a formal reasoning the information flow security in weak memory model and 
the method we will develop will be applied to verify the security of seL4-based 
software for critical embedded devices.  The project is still in nascent phase; however, 
Toby and his team  has developed a software tool, SecC, to formally reason about information flow in 
concurrent C programs.  I have used this tool to develop a formally verified  differentially private location server
which masks the original location of a user without hampering the utility 
and an auction server which does not leak any information classified as secret.  We are in the process 
of submitting this paper to IEEES\&P.  

During my PhD at the Australian National University, Canberra,  I addressed three main challenges of
electronic voting: correctness, privacy, and verifiability.  I addressed correctness by implementing 
the vote counting algorithm inside Coq theorem prover, privacy concern by using Homomorphic 
encryption, and verifiablity concern by generating zero-knowledge-proofs for various claims.

Finally, I feel that I am perfect fit for this role as my experience and skill set, formal verification, cryptography,  and program verification,
would make me 
a invaluable candidate for this position and to your research group because my expertise  in formal verification would help your 
team in building a formally verified correct method,  very similar to the paper CheckDP published in CCS 2020 but more robust. 
I appreciate your consideration
and look forward to hearing for you.   Please contact me if you need any further information. 


Mukesh Tiwari\\
Research Fellow \\
School of Computing and Information Systems\\
University of Melbourne\\
Melbourne, Victoria\\
https://findanexpert.unimelb.edu.au/profile/860472-mukesh-tiwari\\
\end{letter}

\end{document}


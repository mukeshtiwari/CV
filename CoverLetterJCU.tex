%basic cover letter template
\documentclass{letter}
\usepackage{amssymb,amsmath}
\usepackage{graphicx}

\oddsidemargin=.2in
\evensidemargin=.2in
\textwidth=5.9in
\topmargin=-.5in
\textheight=9in

%\address{Mathematics Department\\University of Illinois\\
%1409 W. Green St\\Urbana, Illinois 61801}

\newcommand {\qed}{\mbox{$\Box$}}
\renewcommand {\iff}{\Longleftrightarrow}
\newcommand {\R}{\mathbb{R}}
\newcommand {\N}{\mathbb{N}}
\newcommand {\Q}{\mathbb{Q}}
\newcommand {\Z}{\mathbb{Z}}

\newcommand {\sub}{\mbox{SB}}


\begin{document}
\begin{letter}{Search Committee\\
Division of Tropical Environments and Societies\\
College of Science and Engineering\\
James Cook University\\
Cairns, Australia}


\opening{Dear Search Committee,}


I am writing to apply for the Lecturer position advertised by the James Cook University, Australia  (17078: Lecturer, Information Technology). 
I am currently working as a Research Fellow with Toby Murray in the School of Computing and Information Systems, University 
of Melbourne, Melbourne. My research area is primarily focussed on formal verification of mission critical software programs.  
Currently, I am working  on Security Concurrent Separation Logic, and the aim of the project is 
 to formally reason about security and information flow for a concurrent program. Moreover, during my PhD at the Australian National 
 University, Canberra, I worked on the formal verification of electronic voting scheme, mainly Schulze method.  I am enthusiastic about 
 contributing to your growing and innovative department, specifically in the area of formal verification and software security. 
 
My graduate research with Dirk Pattinson was to bring electronic voting close to 
paper ballot election.  Since the introduction of secret ballots in Victoria, Australia in 1855, 
paper (ballots) are widely used around the world to record 
the preferences of eligible voters. Paper ballots provide three 
important ingredients: correctness, privacy, and verifiability. 
However, the paper ballot election brings various  other challenges, e.g. 
it is slow for large democracies like India,  and error prone for complex voting method 
like single transferable vote, and poses operational challenges for 
large countries like Australia, specifically areas like Northern Territory. 
In order to solve these problems and various others, 
many countries are adopting electronic voting. However, 
electronic voting has a whole new set of problems. In most cases, the software 
programs used to conduct the election have numerous problems, including, but no limited to, 
counting bugs, ballot identification, etc. Moreover, 
these software programs are treated as commercial in confidence and 
are not allowed to be inspected by general member of general public. 
As a consequence, the result produced by these software programs 
can not be substantiated. I answered the three main concerns posed by electronic voting, i.e. 
correctness, privacy, and verifiability. I addressed the correctness concern by using 
theorem prover to implement the vote counting algorithm, 
privacy concern by using homomorphic encryption, and verifiability concern 
by generating a independently checkable scrutiny sheet (certificate). 
My research goal is to keep pushing  the boundaries and continue 
the further development of secure electronic voting, specifically focussing on correctness, privacy, and verifiability. 
Moreover, I would carry on the collaboration with Dirk Pattinson, Rajeev Gore, and Thomas Haines and 
local faculty at the James Cook University, having the 
expertise in cryptography, specifically homomorphic encryption, zero-knowledge-proof, multi-party computation.

 
As a Research Fellow at the University of Melbourne, I am investigating the 
information flow security in weak memory model in presence of concurrency with Toby Murray. 
Verifying the correctness of functional programs without any side effect in a theorem prover 
is not very difficult. Moreover, in most of cases, the proofs are simple pen-and-paper proof. 
However, in case of side effects, the reasoning about the correctness of program is
difficult, mainly in the context of heap manipulating programs, and this situation gets 
more complicated in the presence of concurrency in weak-memory. Our project is geared 
towards a formal reasoning about information flow security and 
the method we will develop will be applied to verify the security of seL4-based 
software for critical embedded devices. It is still in nascent phase; however, our 
ultimate goal is to add the  information flow reasoning logic for C/C++11, formally 
prove it in Coq theorem prover, and develop a tool to automate it. In future, I would 
continue working with Toby Murray and seL4 team.   


In addition to my research, I have facilitated the success of students through teaching and mentoring.  
I would be delighted to use these skills to enhance the success of your university in developing  
future generation leaders in security. In addition, I believe, my research would contribute to 
your department's goal in engagement with industry, community agencies and
professional bodies.  In long term, I want to expand my field of study to privacy preserving 
machine learning,  using fully homomorphic encryption to perform computation and its 
use in healthcare data, designing domain specific languages to automate the security policies 
of an organization, and using multi-party computation to design electronic voting schemes. 


Please contact me if you need any further information. 
Thank you for your consideration.


\closing{Sincerely,
\includegraphics[width=6cm]{mukeshsign.JPG}
}
%I made a pdf file of my signature using the scanner, so that all the cover letters I sent
%electronically were "signed."  

Mukesh Tiwari\\
Research Fellow \\
School of Computing and Information Systems\\
University of Melbourne\\
Melbourne, Victoria\\
mukesh.tiwari@unimelb.edu.au\\
https://findanexpert.unimelb.edu.au/profile/860472-mukesh-tiwari\\

\end{letter}

\end{document}


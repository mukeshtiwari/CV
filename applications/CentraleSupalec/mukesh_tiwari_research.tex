\documentclass[a4paper]{article}
\title{Research Statement}
\author{Mukesh Tiwari}
\date{}
\usepackage{url}
\setlength{\topmargin}{-10mm}
\setlength{\textwidth}{7in}
\setlength{\oddsidemargin}{-8mm}
\setlength{\textheight}{9in}
\setlength{\footskip}{1in}


\begin{document}
\fontsize{12}{15}
\selectfont
\maketitle


My research interests lie in constructing correct software programs used in a democratic process, e.g., 
voting. During my PhD, I worked on formal verification of the Schulze method 
--a preferential vote-counting method-- \cite{10.1007/978-3-319-66107-0_26, bennett2017no} in the Coq theorem prover. 
In my Coq implementation, I assumed that (preferential) ballots were in plaintext, but preferential ballots 
admit ``Italian'' attack \cite{Otten, Benaloh:2009:SSC}. 
Although my previous two Coq implementations \cite{10.1007/978-3-319-66107-0_26, bennett2017no} 
satisfied correctness and verifiability criteria, they lacked privacy due to ``Italian'' attack. 
Therefore, to avoid the ``Italian'' attack on the Schulze method, I used a homomorphic encryption 
to encrypt the ballots and computed the winner by combining all the 
(encrypted) ballots, without decrypting any individual ballot (privacy) \cite{10.1007/978-3-030-41600-3_4}.
The downside of an encrypted ballot Schulze election, 
or in fact any encrypted ballot election, is difficulty in auditing because of involved 
mathematics of cryptography. Therefore, in the future I want to explore the possibility of a 
verified prototype of scrutiny-sheet checker for encrypted ballots Schulze elections.
Finally, I worked on a verified prototype of a simple approval voting election scrutiny-sheet checker for
International Association of Cryptologic Research (IACR) \cite{10.1145/3319535.3354247}. 
In addition, I was involved in the formalisation of single transferable vote, used in the Australian Senate
\cite{10.1007/978-3-030-00419-4_4}.

My long-term aim is to make formal verification ubiquitous in 
software development, specifically for the software programs deployed in public domain
that affect common people.
My expertise in \textbf{Theorem Proving, Cryptography, and Election Security}
gives me an unique perspective to solve challenging problems that matter to many democracies 
and its citizens. 

\section{Research Integration Project}
	Throughout my research career, I have secured softwares by proving them 
	mathematically correct using the Coq theorem prover, and 
	at CentraleSup\'elec (Rennes campus) I would like to use my formal 
	verification expertise to strengthen the capability of \textbf{CIDRE} team in 
	formal methods for security. For example, I can be contribute to 
	the ongoing formal verification projects, e.g., FreeSpec, etc. 
	In addition, I am open to explore other areas of security, e.g.,
	network security, malware analyses, embedded system security, etc.
	


\section{Future Work}
	In addition to contributing to the existing projects at \textbf{CIDRE}, 
	I would like to work on following projects to add more capability 
	at \textbf{CIDRE} and forge research collaboration with other 
	research groups in France and worldwide.

\subsection{Mathematically Proven Correct Cryptographic Algorithms}
	Cryptographic algorithms are used ubiquitously to secure 
	the data, and correctness is an utmost requirement for 
	any cryptographic algorithm implementation. Therefore, 
	I will focus on developing mathematically proven correct cryptographic 
	algorithms used in electronic voting, blockchain, and secure communication, e.g., 
  	sigma protocols (zero-knowledge-proof), verifiable (shuffling) mix-networks, 
  	multi-party computations, secret sharing, zk-snark, etc.
	The rationale behind implementing these algorithms is that anyone can use them to construct 
	an utility, e.g., an election scrutiny-sheet checker, 
	a vote-tallying system based on blockchain,
	a verifiable ballot mixing service, an auction server, etc. One of the motivation
	behind this project is to replace the SwissPost Java implementations\footnote{\url{https://bit.ly/3EODmnF}} 
	with mathematically proven correct Coq 
	implementations\footnote{An ongoing project \url{https://github.com/mukeshtiwari/Dlog-zkp/}}.

	

	

\subsection{Mathematically Proven Correct Vote-Counting Algorithms}
In future, I will focus on developing mathematically proven correct
software programs for vote-counting methods used across the world
such as \textit{Single Transferable Vote (STV), 
First Past the Post (FPTP), Instant-runoff voting(IRV), etc.,} in the Coq theorem prover. 
All the vote-counting methods, by design, lend themselves well to computing the 
winner from plaintext ballots; however, so far there is very little research in computing 
the winner from encrypted ballots, while ensuring correctness, privacy, and 
verifiability. Therefore, producing the winner from encrypted 
ballots is a challenging task.
The motivation for this project is that once we have  mathematically proven correct 
components, anyone --election commission or members of general public-- can use them 
to conduct elections, referendums, and verify elections' outcome 
without worrying about software bugs.  The cryptographic algorithms formalised in 
the previous step are going to be used as a building block in this project.



 

\subsection{Mathematically Proven Correct Decentralised Application}
I will focus on a mathematically proven correct decentralised peer-to-peer technical solution 
\cite{liu2004linkable, Clarke2001, schimmer2009peer, 10.1145/1866307.1866346} in the Coq theorem prover. The motivation 
is to help 
whistleblowers in leaking documents and exposing corruption without revealing their identities.
Being vocal against the government is one the most fundamental right of any citizen, but many 
authoritative governments do not appreciate dissent of any form. Therefore, it uses 
its powerful machinery to punish dissidents, in the name of national security. 
The inspiration for this project comes from David McBridge\footnote{\url{https://en.wikipedia.org/wiki/David_McBride_(whistleblower)}} and 
Richard Boyle\footnote{\url{https://bit.ly/3OQ6kbC}}.
David McBride 
is facing a threat of lifetime jail after
leaking the material alleging war crimes by members of the Australia's Special Operations
Task Group in Afghanistan, while Richard Boyle is facing 161 years for exposing the corruption 
inside the Australian Taxation office
(Australia is ranked very high in 
democracy index\footnote{\url{https://worldpopulationreview.com/country-rankings/democracy-countries}}). 
This research will open the door of collaboration with many groups working in verified 
networking, and verified distributed systems. 

\subsection{Mathematically Proven Correct Social Choice Properties}
Computational social choice theory is a research area that is concerned
with aggregation of ballots (preferences)  of multiple voters (agents) and encompasses 
computer science, mathematics, economics, and political science. Typical applications of 
computational social choice theory is voting (preference aggregation), resource allocation, and fair division.
Most of the proofs in computational social choice theory are pen-and-paper proofs, 
and one of my long term future research goal is to make them more precise using the Coq theorem prover \cite{tiwari2021machine}.
Moreover, I would also focus on designing voting algorithms.
This research opens the door of collaboration with political scientists, 
social choice theorists, economists, and game theorists.


\bibliography{research-statement-bibliography} 
\bibliographystyle{unsrt}

\end{document}
\documentclass[a4paper]{article}
\title{Achievement}
\author{Mukesh Tiwari}
\date{}
\usepackage{url}
\setlength{\topmargin}{-10mm}
\setlength{\textwidth}{7in}
\setlength{\oddsidemargin}{-8mm}
\setlength{\textheight}{9in}
\setlength{\footskip}{1in}
 
\begin{document}
\fontsize{12}{15}
\selectfont
\maketitle


I am the first researcher to develop a method for the Schulze method 
to count encrypted ballots. In addition, I have formalised my method 
in the Coq theorem prover  and extracted an OCaml code that was 
able to produce the result for a small election, 10000 encrypted ballots, 
in reasonable amount of time. This work was particularly very challenging 
because it involved many concepts, e.g., zero-knowledge-proof, 
mix-network, that were needed to ensure the universal verifiability 
of an election. Even though these concepts are very familiar in 
cryptography, they have never been formalised in the Coq theorem 
prover with the purpose to extract an OCaml code, that can be used 
in a real world election. Therefore, I had to figure out all the 
details from scratch to ensure the universal verifiability. Interestingly, 
many researchers not using a theorem prover leave the details, but 
in my case, I needed to flesh down every single detail. In addition, 
I am  the first researcher, with my other research collaborators,
to write a formally verified scrutiny sheet checker for 
the International Association for Cryptologic Research (IACR) election and 
Swiss election (submitted to USENIX). Our scrutiny sheet checker is the first formally verified 
software program to establish software independence, coined by Ron 
Rivest.




\end{document}
%Author by Rajib Das Bhagat (rajibdasbhagat@gmail.com)
%
\documentclass[11pt,a4paper,roman]{moderncv}      
\usepackage[english]{babel}

\moderncvstyle{classic}                            
\moderncvcolor{black}                            

% character encoding
\usepackage[utf8]{inputenc}
\usepackage{url}

% adjust the page margins
\usepackage[scale=0.90]{geometry}

% personal data
\name{Mukesh Tiwari}{}
\email{mt883@cam.ac.uk}
\phone[mobile]{+447824648138}               
\address{Cambridge, United Kingdom}


\begin{document}

\recipient{To}{Dr. Surya Nepla \\
               Group Leader, \\CSIRO's Data61 \& Deputy Research Director Cybersecurity}
\date{\today}
\opening{\textbf{Application for the post of Senior Research Scientist - Cyber Security (81881)}}
\closing{Your Sincerly, \vspace{-1em}}



\makelettertitle



Dear Dr. Surya Nepal, 
\\
%references such as what and how you got this information
\vspace{1em}
I am writing to apply
the job \textbf{Senior Research Scientist - Cyber Security}. 
I have an extensive experience in
security research, in particular cryptography and election security. 
I have a PhD from the Australian National University, Canberra
and Currently, I am working as a Senior Research Fellow at the University of 
Cambridge since October 2021. Before moving to Cambridge, I was a 
research fellow at the University of Melbourne. 

\vspace{0.5cm}
The goal of my PhD was to 
bring  three important ingredients, correctness, privacy, and (universal) verifiability, of a 
paper ballot election to an electronic setting (electronic voting). I 
demonstrated the correctness by implementing and proving the correctness of 
a vote-counting algorithm, the Schulze method, in the Coq theorem prover, 
privacy by using homomorphic encryption to encrypt the ballots and computed
the winner by combining all the (encrypted) ballots, and 
verifiability by means of various zero-knowledge-proofs, the validity of which can
be independently substantiated by an election scrutineer.
At CSIRO, as a Senior Research Scientist, 
I would like to expand my research area into formal verification 
of quantum algorithms, cryptography used in Internet-of-Things, etc. 
My research has been published Interactive Theorem Proving (ITP), 
Computer and Communications Security (CCS), Electronic Voting (EVote), 
and various other conferences. 

 

\vspace{0.5cm}
As a senior research associate at the University of Cambridge, 
I am working on  
a mathematical correct-by-construction framework in the Coq theorem prover 
based on theory of routing algebra in \emph{semiring} algebraic structure.
The goal is to alleviate network-engineers from proving the 
correctness and security of their (network) protocol and focus entirely on protocol design.
All  they need to do is express their protocol in my (mathematical) 
framework, and it will 
tell what property the protocol follows and what it does not. 
In addition, in our framework, depending on concrete instantiation 
of semiring operators, the same algorithm can compute shortest path, longest paths, 
widest paths, multi-objective optimisation, data flow of imperative programs, and many more. 


As a research associate at the University of Melbourne, I did
 acquire hands-on knowledge of separation logic and information flow
 security. I have spearheaded three projects:
 (i) A formally verified auction server, (ii) A formally
 verified location server, and (iii) A formally verified machine learning 
 algorithm that is resistant to side-channel attacks and can  
 run inside the Intel SGX (Software Guard Extensions) enclave for learning
 on sensitive data. All three implementations have been proven 
 memory safe (using separation logic) and free from information
 leaks using the SecureC tool, a tool developed at the University of Melbourne.

\vspace{0.5cm}
I look forward to hearing from you. Let me know if you have any questions. \\
 

\vspace{0.5cm}


\makeletterclosing

\end{document}


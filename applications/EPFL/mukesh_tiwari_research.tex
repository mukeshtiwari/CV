\documentclass[a4paper]{article}
\title{Research Statement}
\author{Mukesh Tiwari}
\date{}
\usepackage{url}
\setlength{\topmargin}{-10mm}
\setlength{\textwidth}{7in}
\setlength{\oddsidemargin}{-8mm}
\setlength{\textheight}{9in}
\setlength{\footskip}{1in}


\begin{document}
\fontsize{12}{15}
\selectfont
\maketitle

Dear Hiring Committee, 

I am writing to apply the job \textit{Faculty Positions in Computer and communications 
Sciences}\footnote{\url{https://facultyrecruiting.epfl.ch/auth/Apply/0/Position/40599568/}}. My diverse skills --formal verification, electronic voting, cryptography, security,
and social choice theory-- makes me an unique candidate for this position.  
My work aims to build \textbf{correct software programs} used in a democratic processes 
using the Coq theorem prover. In particular, I focus on formal verification of 
software programs used 
in elections (electronic voting), cryptography, and computational social choice theory. 

In recent years, many entities of a government are using software programs 
in critical decision making that impacts members of general public,  e.g., counting 
(electronic) ballots to produce the winner of an election. However, if the
software programs contain bugs, then it may produce a wrong 
output, and therefore, government entities can lose its reputation. 
In addition, it can also hamper the trust of  
members of general public in government decision-making processes.
Therefore, it is more imperative than ever  that  we formally verify these software programs 
to make the government decision-making more trustworthy. 



Fair elections are the only way to keep a democracy alive. Most countries (election commission) use paper ballots to record 
its citizens' votes (preferences). Election commission then produces the final tally from these votes, 
while many scrutineers --workers of the participating political parties and members of general 
public in some countries, e.g., France-- observe the counting process. 
These scrutineers and paper ballots, both, are crucial to ensure that 
the election is free from fraud. Scrutineers keep the counting process transparent by making sure all the vote counting 
rules are followed, while paper ballots ensure verifiability because in case of a dispute these paper ballots 
can be counted again. Many countries, however, in recent years are introducing computers to conduct some part, or all, of 
election processes because it is cost effective, accessible to disabled voters, faster result, convenient, etc. 
In addition, some voting methods are very complex to count by hand, e.g., 
single transferable vote, used in Australian senate elections, and it may take months to declare the final result if 
counted by hand. Therefore, the Australian election commission scans all the ballots of a senate election and 
uses a (closed source) software program to produce the final tally. In addition, several cantons in Switzerland use an electronic 
voting system developed by Swiss Post, India uses electronic voting machines, various municipalities in France 
use electronic voting machines, etc.

\section{PhD Work}
My PhD research was focused on verifying electronic voting, specifically vote-counting schemes, in 
the Coq theorem prover. The goal was to 
bring  three important ingredients, correctness, privacy, and (universal) verifiability, of a paper ballot election to 
an electronic setting (electronic voting). In a paper ballot election, correctness is
ensured by scrutineers, and privacy and verifiability  
come for free because of secret paper ballots.  However, achieving these three desirable properties 
are difficult in 
electronic voting because software programs, used in 
various stages of an election, work in a opaque (blackbox) manner. In this (electronic) setting, 
a software program produces the final tally from the cast ballots without any human (scrutineers) involvement, 
other than pressing some 
buttons to run the software program. This opacity can cause harm of various level, including electing a 
wrong winner that is not intended by the voters but because 
of software bugs\footnote{A software bug elected a wrong winner: \url{https://www.areanews.com.au/story/3971893/mercuri-robbed/}}. Most of 
these software programs used across the world by various nations to conduct elections for a public office 
lacks quality measures \cite{10.1145/3014812.3014837, 9152765}. More importantly, these software programs are 
treated as commercial in confidence and therefore,
the members of general public are not allowed to inspect these software programs \cite{AEC:2013:LMM}.





In my thesis, I demonstrated the correctness of a vote counting software program 
by implementing the Schulze method and proving its correctness in Coq theorem 
prover \cite{10.1007/978-3-319-66107-0_26}. 
The Schulze method is a preferential (ranking) voting method where voters rank the participating 
candidates according to their preferences. It is one of the most popular voting method amongst the open-source projects and 
political groups\footnote{\url{https://en.wikipedia.org/wiki/Schulze_method#Users}}.
While no preferential voting scheme can guarantee all
desirable properties that one would like to impose due to Arrow's impossibility theorem \cite{arrow1950difficulty}, 
the Schulze method offers a good compromise with a number of important properties established by economists, 
social choice theorists, and political scientists.  In addition, my implementation 
ensured (universal) verifiability by producing a scrutiny sheet 
with the winner of an election. The scrutiny sheet contained all the data related 
to the election, that could be used to audit the election independently. 
The extracted OCaml code from this formalisation, however, was 
very slow, so I wrote a fast implementation, that I proved equivalent to the slow one.
The fast implementation was able to count millions of ballots \cite{bennett2017no}.
In both formalisation, I assumed that (preferential) ballots were in plaintext, i.e., 
ranking on every ballot was in a (plaintext) number.  Preferential ballots, 
however, admit ``Italian'' attack \cite{Otten, Benaloh:2009:SSC}. 
If the number of participating candidates are significantly high in 
a preferential ballot election,
then a ballot can be linked to a particular voter if published on a bulletin board.
A coercer demands the voter to mark them as first and for the rest of candidates
in a certain permutation. Later, the coercer checks if that permutation appears 
on the bulletin board or not. In order to
avoid this attack on the Schulze method, I used homomorphic encryption to count the (encrypted) ballots, without decrypting 
any individual ballot. Moreover, I addressed verifiability by generating a scrutiny sheet (certificate) 
augmented with zero-knowledge-proofs for various claims, e.g., honest decryption, honest shuffle,  
during the counting \cite{10.1007/978-3-030-41600-3_4}. 
Finally, I wanted to develop to formally 
verified scrutiny sheet checker for encrypted ballots Schulze election, but due to the lack of 
time\footnote{PhD duration is 3.5 years in Australia} I worked on a scrutiny sheet checker for a simple approval voting election,
International Association of Cryptologic Research (IACR) election \cite{10.1145/3319535.3354247}.
In addition, I was involved in formalisation of single transferable vote, used in the Australian Senate
\cite{10.1007/978-3-030-00419-4_4}.




\section{Current Work at Cambridge and Previous Work at Melbourne}
Currently, I am working as a postdoctoral researcher at the university of Cambridge. 
In my current project \emph{Combinators for Algebraic Structures}\footnote{\url{https://www.cl.cam.ac.uk/~tgg22/CAS/}}, 
I am formalising various graph algorithms on \emph{semiring} algebraic 
structure and combinators (functions) to 
combine two, or more, algebraic structures. In this work, I am developing 
a mathematical correct-by-construction \cite{10.1007/978-3-319-66107-0_26} 
framework, in Coq thereom prover, based on theory of routing algebra 
\cite{10.1093/imamat/15.2.161, 10.1145/1080091.1080094}. The goal is to alleviate network-engineers from 
proving the correctness of their protocol and allow them to focus entirely on protocol design.
All they need to do is express their protocol in our
framework, and it will tell what properties the protocol follows and what it does not. 
In addition, this framework can be used in operation
research, given that its underlying principles are 
very similar to protocol design.



At the university of Melbourne, I worked on 
security concurrent separation logic for formally reasoning about the information flow in concurrent programs. 
I used SecureC, a tool developed at the university of Melbourne, to formalise an email server, 
an auction server, and a location server. All these works were proven to leak no sensitive 
information to attackers, assuming that the compiler respects all assumption\footnote{under submission}. 
In addition, I developed a information flow secure gradient descent algorithm in SecureC for 
trusted execution environment, e.g., Intel SGX and ARM TrustZone. This work has been 
informally presented at PaveTrust
\footnote{\url{https://www.acsac.org/2021/workshops/pavetrust/PAVeTrust-Program.pdf}}.


\section{Future Work}
My long-term aim is to make formal verification accessible and ubiquitous in 
software development, specifically for the software programs deployed in public domain
that affect common people.
My expertise in \textbf{Theorem Proving, Cryptography, and Election Security}
gives me an unique perspective to solve challenging problems that matter to many democracies 
and its citizens. In future, I will:

\begin{itemize}



\item focus on formally verified cryptographic primitives used in electronic voting, 
  Internet of Things (IoT), and blockchain, e.g., 
	sigma protocols (zero-knowledge-proof), verifiable (shuffling) mix-networks, 
	multi-party computations, secret sharing, secure communication, zk-snark, etc. 
	In some of my projects, I have formalised
	cryptographic primitives but ended up extracting OCaml 
	code\footnote{We can extract OCaml/Haskell/Scheme 
	code from Coq formalisation.} and used OCaml compiler to 
	compile the code to machine level. However, OCaml compiler is not proven correct 
	and therefore it may introduce new bugs in the compiled code. The challenging part 
	of this project will be to compile a formalised Coq cryptographic 
	code to an assembly code. It is highly non-trivial, but it opens the door of 
	collaboration with other research group working in verified compilation.

\item focus on developing formally verified (electronic) voting 
software (components) programs in Coq theorem prover. 
The rationale is that once we have formally verified 
components, anyone --election commission or members of general public-- can use them 
to conduct elections, referendums, and verify elections' outcome.
It will be interesting to design and implement electronic voting components  
for vote counting methods such as \textit{Single Transferable Vote (STV), 
First Past the Post (FPTP), Instant-runoff, etc.} on encrypted ballots. These 
voting methods differ from each other so producing final tally from encrypted 
ballot will be a challenging task, while ensuring correctness, privacy, and 
verifiability. The outcome of this project can be used by election commissions and citizens of 
the countries that are using software programs to conduct elections, e.g.,
Australia, Switzerland, Canada, France, etc. 

\item focus on formally verified decentralised peer-to-peer technical solution, inspired by 
\cite{liu2004linkable, Clarke2001, schimmer2009peer}, in Coq theorem prover which will help 
whistleblowers in leaking documents and exposing corruption without revealing their identity. 
Being vocal against the government is one the most fundamental right of any citizen, but many 
authoritative governments do not appreciate dissent of any form. Therefore, it uses 
its powerful machinery to punish the dissident, in the name of national security. 
For example, David McBridge\footnote{\url{https://en.wikipedia.org/wiki/David_McBride_(whistleblower)}}, 
a former Australian Defence Force lawyer,  
is facing a threat of lifetime jail after
leaking the material alleging war crimes by members of the Australia's Special Operations
Task Group in Afghanistan (Australia is ranked very high in 
democracy index\footnote{\url{https://worldpopulationreview.com/country-rankings/democracy-countries}}). 
This research will open the door of collaboration with many groups working in verified 
networking, and verified distributed systems. 

\item focus on formally verified (computational) social choice theory in Coq theorem prover. 
	Voting methods admit many properties established by political scientists, social choice theorists, 
	and economists. For example, the Schulze method follows Condorcet criterion, reversal symmetry,
	polynomial runtime, etc., so when we formalise the Schulze method, or in fact any vote-counting method, 
	we can push the boundary of correctness by proving that our 
	implementation of the Schulze method also follows all the properties \cite{tiwari2021machine}. 
	In addition, we can also analyse these voting
	methods from computational complexity point of view whether by a certain amount of bribing
	voters a specified candidate can be made the election's winner \cite{faliszewski2006complexity},
	also by formalising them in the Coq theorem prover. 
	This research opens the door of collaboration with political scientists, 
	social choice theorists, economists, and game theorists.

\item focus on formally verified combinators for algebraic structure (CAS) in Coq theorem prover. 
	Currently, CAS formalisation is highly focused on networking protocols, but it can be 
	adapted for other areas, e.g., optimisation, clustering, algebraic program 
	analysis, etc. In the CAS, we use an abstract algebraic structure 
	\emph{semiring} as an underlying structure
	for computation and and we model various algorithm at the top this 
	structure. In this setting, an algorithm can compute different values depending 
	on the concrete structure of semiring, e.g., the same algorithm 
	can compute shortest path, longest paths, 
	data flow of imperative programs, and many more \cite{gondran2008graphs} 
	for an appropriate semiring.
	

\end{itemize}

My lab will be a very diverse place where students and researchers from formal 
verification, cryptography, 
political science, social science, social choice theory will interact, discuss, 
collaborate on the ideas that matters to democracies and societies.


\bibliography{research-statement-bibliography} 
\bibliographystyle{unsrt}

\end{document}
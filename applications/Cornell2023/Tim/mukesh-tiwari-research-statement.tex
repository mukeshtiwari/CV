\documentclass[a4paper]{article}
\title{Research Statement}
\author{Mukesh Tiwari}
\date{}
\usepackage{url}
\setlength{\topmargin}{-10mm}
\setlength{\textwidth}{7in}
\setlength{\oddsidemargin}{-8mm}
\setlength{\textheight}{9in}
\setlength{\footskip}{1in}
\linespread{0.90} 

\begin{document}
\fontsize{11}{15}
\selectfont
\maketitle

My work aims to build \textbf{correct software} 
using the Coq theorem prover. 
I focus on formal verification of computer programs used in elections, cryptography, 
networking, and computational social choice theory. Below are some potential projects 
that I would like to focus, if given the position.


\section{Future Work}
\begin{itemize}
    \item I would like to focus 
	on formally verified cryptographic primitives used in
	verifiable computing, multi-party computation (MPC), and blockchain. 
    The recent development in the 
	area of zero-knowledge-proofs, in particular 
	zero-knowledge succinct non-interactive argument of knowledge (zk-SNARK), 
	has led to a exponential growth in many useful 
	privacy preserving applications in verifiable computing, MPC, etc. Most of these 
	applications, however, are written 
	in unsound language (Rust/Python/C/Java) and contain bugs,
	lack security proof, and therefore using formal verification 
	to improve their correctness and security would bolster the 
	confidence of a user of these applications.
	In addition, zk-SNARK has massive potential in electronic-voting and machine learning. 
	\begin{itemize}
		\item The current practice to verify the integrity of an election conducted electronically is to 
		recompute the whole count on the publicly available (electronic) ballots, but 
		this excludes many voters from verifying the integrity of the election because 
		they do not posses a powerful enough computing device. Many of 
		zk-SNARKs produce a very small size proof for the integrity of a computation 
		regardless of the computation data. Therefore, if we deploy zk-SNARK in 
		electronic-voting, it would 
		increase the number of scrutineers because anyone, including the 
		mobile devices holders, can verify the integrity of an election 
		by checking these small size proofs. 
		\item In recent years, 
   machine learning models are getting bigger and bigger and cannot be trained using a normal computer. 
   Therefore, most of them are trained in cloud, but this 
   raises the question if the cloud has used the right set of data to train them model or not. 
   One way to solve this problem is to use verifiable computing and force the cloud to generate a short proof (zk-SNARK)
   with the trained machine learning model. 
   This short proof can be then checked by any independent third party to attest the 
   integrity of training. 


	\end{itemize}



\item At the University of Melbourne, I worked on 
security concurrent separation logic for formally reasoning about the information 
flow security in concurrent programs. I used \textit{SecureC}, a tool developed at the university of 
Melbourne, to formalise an email server, an auction server, a location server, and 
a differentially private gradient descent algorithm. 
All these works were proven to leak no sensitive 
information to attackers, assuming that the compiler respects all 
assumption and soundness of implementation of \textit{SecureC}. However, 
\textit{SecureC} is very limited in terms of expressibility and tooling 
and therefore it cannot be used to model a real world case study. I would like to work on 
adding information flow support in Iris --a Coq library-- with my co-author Toby Murray. 


\item In one of my recent project, me with my co-authors --after 1.5 years of 
painstaking effort-- formalised a critical 
piece of code (mix-network) in the SwissPost --used in legally binding elections 
in Switzerland-- in the Coq theorem prover and extracted an OCaml code from our 
formalisation against which the Java code of SwissPost can be tested. 
In the process, we found a flaw in a decade old cryptographic proof.  Given my 
expertise in formal verification I would like, with my PhD students and 
postdocs researchers, to explore and formalise other 
piece of code used in public domain, e.g., machine learning 
models deployed in self-driving cars, decision-making algorithms, etc. 

\end{itemize}	

\section{Past Work}
My PhD research was focused on verifying electronic voting, specifically vote-counting schemes, in 
the Coq theorem prover. The goal was to 
bring  three important ingredients (i) correctness, (ii) privacy, and (iii) verifiability of a 
paper-ballot election to 
an electronic setting (electronic voting).
In my thesis, I demonstrated the correctness of the Schulze method 
by implementing it and proving its correctness in the Coq theorem 
prover. The Schulze method is a preferential (ranking) voting method where voters rank the participating 
candidates according to their preferences. It is one of the most popular voting method amongst the open-source projects and 
political groups\footnote{\url{https://en.wikipedia.org/wiki/Schulze_method#Users}}.
In addition, my implementation 
ensured (universal) verifiability by producing a scrutiny sheet 
with the winner of an election. The scrutiny sheet contained all the data related 
to the election that could be used to audit the election independently. 
The extracted OCaml code from this formalisation, however, was 
very slow, so I wrote another fast implementation, proven equivalent to the slow one,
capable of counting millions of ballots.

In both formalisation, I assumed that (preferential) ballots were in plaintext, i.e., 
ranking on every ballot was in a (plaintext) number.  Preferential ballots, 
however, admit ``Italian'' attack. 
If the number of participating candidates are significantly high in 
a preferential ballot election,
then a ballot can be linked to a particular voter if published on a bulletin board.
A coercer demands the voter to mark them as first and for the rest of candidates
in a certain permutation. Later, the coercer checks if that permutation appears 
on the bulletin board or not. In order to
avoid this attack on the Schulze method, I used homomorphic encryption to count the (encrypted) ballots, without decrypting 
any individual ballot. Moreover, I addressed verifiability by generating a scrutiny sheet (certificate) 
augmented with zero-knowledge-proofs for various claims, e.g., honest decryption, honest shuffle,  
during the counting. 
Finally, I wanted to develop to formally 
verified scrutiny sheet checker for encrypted ballots Schulze election, but due to the lack of 
time\footnote{PhD duration is 3.5 years in Australia} I worked on a scrutiny sheet checker for a simple approval voting election,
International Association of Cryptologic Research (IACR) election.
In addition, I was involved in formalisation of single transferable vote, used in the Australian Senate election.




My lab will be a diverse place where students and researchers from formal 
verification, cryptography, 
political science, social science, social choice theory will interact, discuss, 
collaborate on the ideas that matters to democracies and common people. 
\end{document}
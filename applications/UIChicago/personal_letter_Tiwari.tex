\documentclass[11pt,a4paper,roman]{moderncv}      
\usepackage[english]{babel}

\moderncvstyle{classic}                            
\moderncvcolor{black}                            

% character encoding
\usepackage[utf8]{inputenc}
\usepackage{url}

% adjust the page margins
\usepackage[scale=0.90]{geometry}
\linespread{0.95} 
% personal data
\name{Mukesh Tiwari}{}
\email{mukesh.tiwari@cs.ox.ac.uk}
\phone[mobile]{+447824648138}               
\address{Oxford, United Kingdom}


\begin{document}

\recipient{To}{The Hiring Committee \\ University of Illinois, Chicago, USA}
\date{}
\opening{\textbf{Application for the tenure-track Assistant Professor (Computer System)}}
\closing{Your Sincerly, \vspace{-1em}}



\makelettertitle


Dear Hiring Committee, 
\\
%references such as what and how you got this information
\vspace{1em}

My name is Mukesh Tiwari and I am a senior research associate at 
the University of Oxford, UK with expertise in formal methods, cybersecurity and privacy, 
and social choice theory. I am writing to apply for the tenure-track \textbf{Assistant Professor} job.
I have extensive research experience in
formal verification (Coq theorem prover), electronic voting, and cryptography, and my research 
experience makes me a valuable candidate for this post.



My research touches the lives of common people and solves 
real-world problems that matter to democracies and common people. For example, my paper (i)
\textbf{Assume but Verify: Deductive Verification of Leaked Information in Concurrent Applications},
accepted in ACM CCS, develops a theory using the information-flow security principals 
for processing sensitive data --ethnic origin, political opinions,
health-related data, and  biometric data-- of common 
people in secure enclave, e.g., Intel SGX, Arm TrustZone, etc. Moreover, 
we demonstrate the usability of our method by developing non-trivial case studies that handles 
sensitive data accompanied by the machine-checked mathematical proofs that none of 
them have unintended side-channel data leakage; 
(ii) \textbf{Machine-checking Multi-Round Proofs of Shuffle: Terelius-Wikstrom and Bayer-Groth}, 
published in USENIX Security, mathematically establishes a critical piece of 
code in the SwissPost voting software --used in legally binding 
elections in Switzerland-- is correct (and debunks a decade old myth of the cryptographic 
community that Terelius-Wikstrom method is zero-knowledge proof. We have formally 
proved in the Coq theorem prover that it is a zero-knowledge argument and not 
a zero-knowledge proof);
(iii) \textbf{Verifiable Homomorphic Tallying for the Schulze Vote Counting Scheme}, published 
in VSTTE, not only developes a publicly verifiable method to count encrypted ballots for 
a complex voting method but it 
is also proven correct in the Coq theorem prover to ensure that there is no gap between 
the pen-and-paper proof and the actual implementation;  (iv) \textbf{Verified Verifiers for 
Verifying Elections}, published in ACM CCS, develops a mathematically proven correct tool 
in the Coq theorem prover to verify the elections conducted by 
the International Association for Cryptologic Research. We have used 
our tools to verify the integrity of IACR elections; (v) \textbf{Modular Formalisation and 
Verification of STV Algorithms}, published in E-Vote, develops a mathematically proven 
correct tool in the Coq theorem prover. We have used this tool to verify
the results of Australian Senate election; 
(vi) \textbf{Verifpal: Cryptographic Protocol Analysis for the Real World}, published in 
INDOCRYPT, develops a tool that can used to model real work cryptographic protocol, and 
Verifpal has been used by many researchers to model security and privacy aspect of 
digital contact tracing during COVID, etc.
At Cambridge, I developed a mathematically proven correct tool in the Coq theorem prover 
that can be used by networking researchers to model networking-protocols in the abstract 
setting of semirings (and we are in the process of submitting our paper in CAV 2024).
At Oxford, I am exploring the avenues to bridge the gap between a security protocol 
(formal communication model) with its implementation using session types, and our 
goal is to produce a more realistic distributed executable model of the security protocol. 
Currently, as a first step, I am focussing on Signal app 
where my goal is to prove (or disprove) that its Java implementation follows the 
communication model described in the Signal's documentation. 
In a nutshell, all my research so far has an impact on the lives on common people and 
researchers. 


Although, as a person, I am slightly introvert, but I firmly believe interdisciplinary
research is the key to solve challenging problem pertaining to society. Therefore, I like to 
chat and work with diverse set of researchers, which is evident from my projects involving 
myriad of concepts, e.g., formal method, cryptography, voting, social choice, 
separation logic, information-flow security, graph theory, session types, etc. 



\vspace{0.5cm}


\makeletterclosing

\end{document}


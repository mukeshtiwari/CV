\documentclass[11pt,a4paper,roman]{moderncv}      
\usepackage[english]{babel}

\moderncvstyle{classic}                            
\moderncvcolor{black}                            

% character encoding
\usepackage[utf8]{inputenc}
\usepackage{url}

% adjust the page margins
\usepackage[scale=0.90]{geometry}
\linespread{0.90} 
% personal data
\name{Mukesh Tiwari}{}
\email{mt883@cam.ac.uk}
\phone[mobile]{+447824648138}               
\address{Cambridge, United Kingdom}


\begin{document}

\recipient{To}{The Hiring Committee \\ The University of Luxembourg, Luxembourg}
\date{}
\opening{\textbf{Application for the post of Assistant Professor in Software Engineering }}
\closing{Your Sincerly, \vspace{-1em}}



\makelettertitle


Dear Hiring Committee, 
\\
%references such as what and how you got this information
\vspace{1em}
My name is Mukesh Tiwari and  I am senior research associate at 
the University of Cambridge, Cambridge, UK. 
I am writing to apply
for the job \textbf{Assistant Professor in in Software Engineering }. 
I have an extensive experience in
formal verification (Coq theorem prover) and security research (Cryptography), and 
I find the 
University of Luxembourg will be  
a perfect place to continue my research in formal verification and expand my horizons in other areas of
software engineering, including AI for software development, program 
synthesis, etc. 


\vspace{0.5cm}
Nowadays, we are relying more and more on software programs for decision making. 
For example, many government entities are making policy decisions based on the 
output of a software program. However, if the software program 
contains bugs, then it may produce a wrong output. Therefore, the government entity 
can lose its reputation. In addition, it can also hamper 
the trust of members of general public in the government decision-making processes. 
Therefore, it is more imperative than ever that we formally verify these software programs 
and develop tools to automate the formal verification process to make the government 
decision-making more trustworthy. 


In my PhD, I have formally verified the Schulze vote counting method. I have addressed ballot privacy by using 
homomorphic encryption, and verifiability by means of producing an independently verifiable scrutiny 
sheet (certificate), the validity of which can be independently substantiated, that witnesses the 
correctness of the execution of an election.  As a research associate at the University of Melbourne, 
I did acquire hands-on knowledge of separation logic and information flow security. I have spearheaded 
two projects: (i) a formally verified auction server and (ii) a formally verified location server. 
Both implementations were proven memory safe (using separation logic) and free from information 
leaks (applying information flow security), using the SecCSL tool. In addition, I also explored side 
channel attacks that might arose due to secret-dependent branching of the code running inside
Intel SGX (Software Guard Extensions) enclave. Currently, in Cambridge, I am working on 
formal verification of graph algorithms on semirings. In this setting, formally verified graph
 algorithm can compute different values depending on the concrete structure of semiring, e.g., the 
 same algorithm can compute shortest path, longest paths, data flow of imperative 
 programs, and many more for an appropriate choice of semiring.


\vspace{0.5cm}
My long term goal is to make formal verification accessible for all kind of 
software programs, and my current goal 
is to push the formal verification for software programs used in 
public domain. I look forward to hearing from you. Let me know if you have any 
questions.\footnote{My research had been severly impacted by Melbourne lockdown and therefore
I was not very productive in year 2020 and 2021.} \\
 

\vspace{0.5cm}


\makeletterclosing

\end{document}


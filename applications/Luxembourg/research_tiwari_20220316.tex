\documentclass[a4paper]{article}
\title{Research Statement}
\author{Mukesh Tiwari}
\date{}
\usepackage{url}
\setlength{\topmargin}{-10mm}
\setlength{\textwidth}{7in}
\setlength{\oddsidemargin}{-8mm}
\setlength{\textheight}{9in}
\setlength{\footskip}{1in}
\linespread{0.90} 
\begin{document}
\fontsize{12}{15}
\selectfont
\maketitle


My work aims to build \textbf{correct software programs} 
using the Coq theorem prover. 
I focus on formal verification of software programs used in elections, cryptography, 
networking, and computational social choice theory. Numerous critical decisions, e.g., 
producing the winner of an election by an election commission,
are taken based on the output of a software program. However, if the
software program contains bugs, then it may produce a wrong 
output. Therefore, the government entity (election commission) can 
lose its reputation.  

My PhD research was focused on verifying electronic voting, specifically vote-counting schemes, in 
the Coq theorem prover. The goal was to 
bring  three important ingredients, correctness, privacy, and (universal) verifiability, of a paper ballot election to 
an electronic setting (electronic voting). In a paper ballot election, correctness is
ensured by scrutineers, and privacy and verifiability  
come for free because of secret paper ballots.  However, achieving these three desirable properties 
are difficult in 
electronic voting because software programs, used in 
various stages of an election, work in a opaque (blackbox) manner. 


My long-term aim is to make formal verification accessible and ubiquitous in 
software development, specifically for the software programs deployed in public domain
that affect common people.
My expertise in \textbf{Theorem Proving, Cryptography, and Election Security}
gives me an unique perspective to solve challenging problems that matter to many democracies 
and its citizens. In future, I will:

\begin{itemize}



\item focus on formally verified cryptographic primitives used in electronic voting, 
  Internet of Things (IoT), and blockchain, e.g., 
	sigma protocols (zero-knowledge-proof), verifiable (shuffling) mix-networks, 
	multi-party computations, secret sharing, secure communication, zk-snark, etc. 
	
\item focus on developing formally verified (electronic) voting 
software (components) programs in Coq theorem prover. 
The rationale is that once we have formally verified 
components, anyone --election commission or members of general public-- can use them 
to conduct elections, referendums, and verify elections' outcome.

\item focus on formally verified decentralised peer-to-peer technical solution, inspired by 
\cite{liu2004linkable, Clarke2001, schimmer2009peer}, in Coq theorem prover which will help 
whistleblowers in leaking documents and exposing corruption without revealing their identity. 



\item focus on formally verified combinators for algebraic structure (CAS) in Coq theorem prover. 
	Currently, CAS formalisation is highly focused on networking protocols, but it can be 
	adapted for other areas, e.g., optimisation, clustering, algebraic program 
	analysis, etc. In this setting, an algorithm can compute different values depending 
	on the concrete structure of semiring, e.g., the same algorithm 
	can compute shortest path, longest paths, 
	data flow of imperative programs, and many more \cite{gondran2008graphs} 
	for an appropriate semiring.
	

\end{itemize}


\bibliography{research-statement-bibliography} 
\bibliographystyle{unsrt}

\end{document}
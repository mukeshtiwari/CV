\documentclass[a4paper]{article}
\title{Teaching Statement}
\author{Mukesh Tiwari}
\date{}
\usepackage{url}
\setlength{\topmargin}{-10mm}
\setlength{\textwidth}{7in}
\setlength{\oddsidemargin}{-8mm}
\setlength{\textheight}{9in}
\setlength{\footskip}{1in}
\linespread{0.85} 
\begin{document}
\fontsize{12}{15}
\selectfont
\maketitle


My teaching philosophy is to not immediately reach a solution but to develop a 
thinking process (problem solving mindset) that leads to the solution. I believe every 
student is different and has a unique style of learning, and my role is to help them find 
and hone their style. When I started teaching as an assistant professor at the International Institute of Information 
Technology (IIIT), Bhubanesware, India\footnote{It is a small technical school},
my single biggest challenge was keeping the students engaged in my class, especially the first year 
students in C programming course. At the IIIT, I taught C programming to 
first year students, Compiler Design and Java programming to third year students, and
Cryptography to final year (4th year) students. In each course, every single 
problem,  more or less, boiled down to keeping the students engaged in a topic. 
In order to keep them engaged in a class, I took a Coursera course on 
learning\footnote{\url{https://www.coursera.org/learn/learning-how-to-learn}} and 
read many academic articles and non-academic articles about effective learning.
I tried some of the techniques suggested in the Coursera course and 
academic and non-academic articles in my classes. Below I 
describe my experiences with teaching and efforts to engage my students. 






In every course, I started with the end goal of the course. For example, 
in C programming course, I told my students that by the end of this 
course they should be 
able to write simple C programs, e.g., calculator, time tracking system, etc. 
The rationale was 
to show a vision to excite them 
for learning and instill the feeling of empowerment, by a narrative from
being a consumer to being a developer of software programs.



To promote active participation, at the beginning of every class I would announce that 
on a few occasions I would make deliberate mistakes in solving a problem, and the
goal of the class was to catch me on the spot.
If the class succeeded, they received  
class participation marks, otherwise I told them my 
mistake and no one received any marks. In every single 
feedback that I got during my 3 years of teaching, 
almost everyone appreciated this idea of making deliberate mistakes. 



For every course I taught, I designed projects related to the concepts 
and ideas present in the course. It was a part of my teaching philosophy to impart 
critical thinking. Furthermore, I asked my students to come up with their own ideas to 
promote idea-exploration, a key step to develop a critical thinking and problem solving mindset. 


I feel qualified to teach most of the computer science courses 
because of my background in computer science, but 
my natural preference is the courses close to my research 
area, e.g., theorem proving, cryptography, logic 
and its applications in computer science, discrete 
mathematics, algorithms and data structures, 
introductory programming course (C/C++/Java/Python/Haskell/OCaml),
foundation of computing, etc. 
In addition, I would also like to design a course to teach 
various voting methods used around the world and 
their pros and cons from a social choice theory perspective. 
Apart from these topics, I will 
be more than happy to teach other courses, given enough 
preparation time, at introductory and 
intermediate levels, e.g., type theory, theory of programming 
languages, networking, operating systems, databases, etc.


\end{document}
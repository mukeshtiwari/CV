\documentclass[a4paper]{article}
\title{Leadership Philosophy}
\author{Mukesh Tiwari}
\date{}
\usepackage{url}
\setlength{\topmargin}{-10mm}
\setlength{\textwidth}{7in}
\setlength{\oddsidemargin}{-8mm}
\setlength{\textheight}{9in}
\setlength{\footskip}{1in}
 
\begin{document}
\fontsize{12}{15}
\selectfont
\maketitle


I have one core value as a leader: help and support people to achieve their best. Everything 
else hinges around it. For example, when I was a teacher in a small technical university 
in India, my goal was not to immediately reach a solution but to develop a 
thinking process (problem solving mindset) that leads to the solution. I believe every 
student is different and has a unique style of learning, and my role is to help them find 
and hone their style. However, I made sure that the class was aware of 
end goals (learning objectives). In addition, I communicated my expectations very clearly
to the class. Therefore, 
in the beginning of every class I would tell the students that what they would be 
able to do after learning the topic, that I was going to teach. 
Clearly communicating my expectations helped my students better understand the end goals
(goal and development orient leadership).

I also like to own the outcome of my decisions as a leader. There were many instances where 
my expectations from the class did not match the reality. I, therefore,  
critically examined my mistakes and revised them. In addition, there were 
many instances during my PhD where some proof ideas did not go the 
way I expected but I learned from them by carefully examining my assumptions. I 
recently realised that communication is one of the key 
factors in effective leadership, but more importantly staying calm 
during a difficult conversation is one the most underrated skill.







\end{document}
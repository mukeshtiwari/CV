\documentclass[11pt,a4paper,roman]{moderncv}      
\usepackage[english]{babel}

\moderncvstyle{classic}                            
\moderncvcolor{black}                            

% character encoding
\usepackage[utf8]{inputenc}
\usepackage{url}

% adjust the page margins
\usepackage[scale=0.90]{geometry}
\linespread{0.90} 
% personal data
\name{Mukesh Tiwari}{}
\email{mt883@cam.ac.uk}
\phone[mobile]{+447824648138}               
\address{Cambridge, United Kingdom}


\begin{document}

\recipient{To}{The Hiring Committee \\ Chalmers University of Technology, Sweden}
\date{}
\opening{\textbf{Application for the post of Assistant Professor}}
\closing{Your Sincerly, \vspace{-1em}}



\makelettertitle


Dear Hiring Committee, 
\\
%references such as what and how you got this information
\vspace{1em}
My name is Mukesh Tiwari and  I am senior research associate at 
the University of Cambridge, Cambridge, UK. 
I am writing to apply
for the job \textbf{Assistant Professor in WASP}. I have an extensive experience in
formal verification (Coq theorem prover) and security research (Cryptography), and I find that the 
Chalmers University of Technology will be  
perfect place to continue my research and expand my horizons in other areas,  
including autonomous systems, AI, networking, constructive type theory, programming 
language, etc.  



\vspace{0.5cm}
I find the goal research in WASP very close to my philosophy of 
mathematically verifying software programs that affect the life of common people. 
We are living in a world where most of the machines are 
controlled by software programs. Moreover, 
it is going to increase in near future, and we 
are going to have more automated machines. However, 
a software program can have bugs and may lead 
to unfortunate situation, possibly killing people. 
For example, recently a few safe technology campaigners
released a disturbing video  
showing self-driving Tesla cars fail to detect 
children (child-sized mannequin) in the road and ended up hitting 
these mannequins\footnote{https://www.theguardian.com/technology/2022/aug/09/tesla-self-driving-technology-safety-children}.
Therefore, it more imperative than ever  that  we formally verify 
all the software programs that interact and affect the common people. 


The reason I am applying at Chalmers is because of its cutting edge research, 
e.g., CakeML by Magnus Myreen, Coq by Thierry Coquand, Lava by Mary Sheeran, 
QuickCheck by John Hughes, and many others. My research in formal verification 
enables me to work with most of the researchers working at WASP, and Chalmers in 
general, but I have found a few researchers\footnote{I can collaborate with 
more than half of the people at Chalmers but due to lack of space, I am listing the people who
are very close to my research.} at Chalmers with whom I can collaborate 
and contribute to WASP projects:
\begin{itemize}
  \item Wolfgang Ahrendt: we can collaborate on blockchain (smart contract) research.
    In addition, we can work on formal certification of Autonomous Driving Software in Coq. 
  \item Aikaterini Mitrokotsa: we can collaborate on formal verification of 
    various cryptographic primitives used in MPC to achieve decentralised
    privacy preserving federated learning.
  \item Magnus Myreen: we can collaborate on designing a formally verified 
    compiler in Coq for a simple language that is  powerful enough for encoding 
    cryptographic primitives, mostly addition and multiplication in prime field.
  \item Mary Sheeran: her research \textit{An Algebra of Array Combinators and its Applications}
   is very similar to what I am currently doing at Cambridge, combinators for 
   network protocol design. We can collaborate on formally verified hardware/software  
   design in the context of automated vehicles for secure communication.
    
\end{itemize}

\section{Reference}
\begin{itemize}
  \item Thomas Haines, Research School of Computer Science, Australian National University, Canberra, thomas.haines@anu.edu.au
  \item V{\'e}ronique Cortier, Research Director, CNRS, LORIA, Nancy, France, veronique.cortier@loria.fr\footnote{I have not worked with V{\'e}ronique, but she understands my work 
  very closely and is very impressed. Therefore, she is very happy to write a letter for me based on just my election security work.}
\end{itemize}


\vspace{0.5cm}
I look forward to hearing from you. Let me know if you have any 
questions.\footnote{My research had been severly impacted by Melbourne lockdown and therefore
I was not very productive in year 2020 and 2021.} \\
 

\vspace{0.5cm}


\makeletterclosing

\end{document}


%Author by Rajib Das Bhagat (rajibdasbhagat@gmail.com)
%
\documentclass[11pt,a4paper,roman]{moderncv}      
\usepackage[english]{babel}

\moderncvstyle{classic}                            
\moderncvcolor{black}                            

% character encoding
\usepackage[utf8]{inputenc}
\usepackage{url}

% adjust the page margins
\usepackage[scale=0.90]{geometry}

% personal data
\name{Mukesh Tiwari}{}
\email{mt883@cam.ac.uk}
\phone[mobile]{+447824648138}               
\address{Cambridge, United Kingdom}


\begin{document}

\recipient{To}{The Hiring Committee, \\
    University of Cambridge,\\ Cambridge}
\date{}
\opening{\textbf{Application for the post of Assistant Professor in Security and Privacy (NR33473)}}
\closing{Your Sincerly, \vspace{-1em}}



\makelettertitle



Dear Hiring Committee, 
\\
%references such as what and how you got this information
\vspace{1em}
I am writing to apply
the \textbf{Assistant Professor} job in Security and Privacy. 
I have extensive research experience in
theorem proving (Coq theorem prover), cryptography, election security, and social choice theory. 
I have a PhD from the Australian National University, Canberra, Australia
and Currently, I am working as a Senior Research Fellow at the University of 
Cambridge since October 2021. Before moving to Cambridge, I was a 
research fellow at the University of Melbourne. 


The goal of my PhD was to 
bring  three important ingredients, correctness, privacy, and (universal) verifiability, of a 
paper ballot election to an electronic setting (electronic voting). I 
demonstrated: (i) correctness by implementing and proving the correctness of 
a vote-counting algorithm, the Schulze method, in the Coq theorem prover, 
(ii) privacy by using homomorphic encryption to encrypt the ballots and computed
the winner by combining all the (encrypted) ballots, and 
(iii) verifiability by means of various zero-knowledge-proofs.
At Cambridge University, as an Assistant Professor, 
I would like to expand my research area into other areas of 
security and formal verification, e.g., domain specific language to reason about 
functional correctness and security properties --from computational complexity perspective-- of 
cryptographic algorithms, anonymous communication, blockchain, zk-snark,  
information flow security, computational complexity of social choice methods, 
etc. My research has been published in Interactive Theorem Proving (ITP), 
Computer and Communications Security (CCS), Electronic Voting (EVote), 
International Conference on Cryptology in India (IndoCrypt),
and some (finished) works have been submitted to ESOP, USENIX, and SIGCOMM. 
All my research work has been formalised in the Coq theorem prover. 

\section{PhD Thesis}
In my thesis, I demonstrated the correctness of a vote-counting software program 
by implementing and proving the correctness of the Schulze method in the Coq theorem 
prover. 
The Schulze method is a preferential (ranking) voting method where voters rank the participating 
candidates according to their preferences. It is one of the most popular voting method amongst the open-source projects and 
political groups.
While no preferential voting scheme can guarantee all
desirable properties that one would like due to Arrow's impossibility theorem, 
the Schulze method offers a good compromise with a number of important properties established by economists, 
social choice theorists, and political scientists. From my Coq implementation of the Schulze method,
I used Coq's extraction mechanism to get an OCaml program.
Then I used the OCaml compiler to compile the extracted OCaml program to get an executable  
to count ballots.
In addition to correctness, my implementation also
ensured (universal) verifiability by producing a scrutiny sheet. The scrutiny sheet contained all 
the data to audit the election independently. The (extracted) OCaml program, however, was 
very slow and could not count more than 10,000 ballots. Therefore, I wrote another 
Coq implementation, proven equivalent to the slow Coq 
implementation, from which the (extracted) 
OCaml program was able to count millions of ballots.

In both, slow and fast, Coq implementations, I assumed that (preferential) ballots were in plaintext, i.e., 
ranking on every ballot was in a (plaintext) number.  Preferential ballots, 
however, admit ``Italian'' attack. 
If the number of participating candidates are significantly high in 
a preferential ballot election,
then a ballot can be linked to a particular voter if published publicly, on a bulletin board.
The attack is: a coercer demands a voter to mark them as first and for the rest of candidates
in a certain given order (permutation). Later, the coercer checks if that the order appears 
on the bulletin board or not. Although my previous two Coq implementations  
satisfied correctness and verifiability criteria, they lacked privacy due to ``Italian'' attack. 
In order to avoid this attack on the Schulze method, I used a homomorphic encryption 
to encrypt the ballots and computed the winner by combining all the 
(encrypted) ballots, without decrypting any individual ballot (privacy). 
Moreover, I addressed verifiability by generating a scrutiny sheet (certificate) 
augmented with zero-knowledge-proofs (zkp), e.g., honest decryption zkp, honest shuffle zkp. 
This work was carried out in the Coq theorem prover, so I ended up achieving
correctness, privacy, and verifiability. The downside of an encrypted ballot Schulze election, 
or in fact any encrypted ballot election, is difficulty in auditing because of involved 
mathematics of cryptography. Therefore, in the future I want to explore the possibility of a 
verified prototype of scrutiny-sheet checker for encrypted ballots Schulze elections.
Finally, I worked on a verified prototype of a simple approval voting election scrutiny-sheet checker for
International Association of Cryptologic Research (IACR). 
In addition, I was involved in the formalisation of single transferable vote, used in the Australian Senate.

\section{Future Research}
My long-term aim is to make formal verification accessible and ubiquitous in 
software development, specifically for the software programs deployed in public domain
that affect common people.
My expertise in \textbf{Theorem Proving, Cryptography, Election Security, and Social Choice Theory}
gives me an unique perspective to solve challenging problems that matter to many democracies 
and its citizens. 	

\subsection{Mathematically Proven Correct Cryptographic Algorithms}
	Cryptographic algorithms are used ubiquitously to secure 
	the data and correctness is an utmost requirement for 
	any cryptographic algorithm implementation. Therefore, 
	I will focus on developing mathematically proven correct cryptographic 
	algorithms used in electronic voting, blockchain, and secure communication, e.g., 
  	sigma protocols (zero-knowledge-proof), verifiable (shuffling) mix-networks, 
  	multi-party computations, secret sharing, zk-snark, etc.
	The rationale behind implementing these algorithms is that anyone can use them to construct 
	an utility, e.g., an election scrutiny-sheet checker, 
	a vote-tallying system based on blockchain,
	a verifiable ballot mixing service, an auction server, etc. One of the motivation
	behind this project is to replace the SwissPost Java implementation, used in democratic elections in 
	Switzerland, with a mathematically proven correct Coq implementation.
	

	

\subsection{Mathematically Proven Correct Vote-Counting Algorithms}
In future, I will focus on developing mathematically proven correct
software programs for vote-counting methods used across the world
such as \textit{Single Transferable Vote (STV), 
First Past the Post (FPTP), Instant-runoff voting(IRV), etc.,} in the Coq theorem prover. 
All the vote-counting methods, by design, lend themselves well to computing the 
winner from plaintext ballots; however, so far there is very little research in computing 
the winner from encrypted ballots, while ensuring correctness, privacy, and 
verifiability. Therefore, producing the winner from encrypted 
ballots is a challenging task.
The motivation for this project is that once we have  mathematically proven correct 
components, anyone --election commission or members of general public-- can use them 
to conduct elections, referendums, and verify elections' outcome 
without worrying about software bugs.  The cryptographic algorithms formalised in 
the previous step are going to be used as a building block in this project.



 

\subsection{Mathematically Proven Correct Decentralised Application}
I will focus on a mathematically proven correct decentralised peer-to-peer technical solution 
 in the Coq theorem prover. The motivation 
is to help 
whistleblowers in leaking documents and exposing corruption without revealing their identities.
Being vocal against the government is one the most fundamental right of any citizen, but many 
authoritative governments do not appreciate dissent of any form. Therefore, it uses 
its powerful machinery to punish dissidents, in the name of national security. 
The inspiration for this project comes from David McBridge and 
Richard Boyle.
David McBride 
is facing a threat of lifetime jail after
leaking the material alleging war crimes by members of the Australia's Special Operations
Task Group in Afghanistan, while Richard Boyle is facing 161 years for exposing the corruption 
inside the Australian Taxation office
(Australia is ranked very high in 
democracy index). 
This research will open the door of collaboration with many groups working in verified 
networking, and verified distributed systems. 

\subsection{Mathematically Proven Correct Social Choice Properties}
Computational social choice theory is a research area that is concerned
with aggregation of ballots (preferences)  of multiple voters (agents) and encompasses 
computer science, mathematics, economics, and political science. Typical applications of 
computational social choice theory is voting (preference aggregation), resource allocation, and fair division.
Most of the proofs in computational social choice theory are pen-and-paper proofs, 
and one of my long term future research goal is to make them more precise using the Coq theorem prover.

My current focus is voting because voting methods admit many excellent (social choice) properties
established by political scientists, social choice theorists, 
and economists. For example, the Schulze method follows Condorcet criterion, reversal symmetry,
polynomial runtime, etc., so when we formalise the Schulze method, or in fact any vote-counting method, 
we can push the boundary of correctness by proving that our 
implementation of the Schulze method also follows all the properties. 
In addition, we can analyse these voting
methods from computational complexity perspective of bribery, if by bribing a certain amount 
voters a specified candidate can be made an election's winner. 
This research opens the door of collaboration with political scientists, 
social choice theorists, economists, and game theorists.

\section{Current Work and Past Work}
In my current project \emph{Combinators for Algebraic Structures (CAS)}, 
I am formalising various graph algorithms on \emph{semiring} algebraic 
structure and combinators (functions) to 
combine two, or more, algebraic structures. In this work, I am developing 
a mathematical correct-by-construction  
framework, in Coq theorem prover, based on theory of generalised path-finding algebra. 
In our framework, depending on concrete instantiation 
of semiring operators, the same algorithm can compute shortest path, longest paths, 
widest paths, multi-objective optimisation, data flow of imperative programs, and many more. 
In fact, the Schulze method is one instance of our framework. However, 
the current CAS implementation is highly focused towards networking protocols,
so as a future work I will focus on adding more algorithms in CAS related 
to multi-objective optimisation, data flow analysis of imperative programs, and voting. 




 As a research associate at the University of Melbourne, I did
 acquire hands-on knowledge of separation logic and information flow
 security. I have spearheaded three projects:
 (i) A formally verified auction server, (ii) A formally
 verified location server, and (iii) A formally verified machine learning 
 algorithm that is resistant to side-channel attacks and can  
 run inside the Intel SGX (Software Guard Extensions) enclave for learning
 on sensitive data. All three implementations have been proven 
 memory safe (using separation logic) and free from information
 leaks using the SecureC tool, a tool developed at the University of Melbourne.



 \section{Teaching}
My teaching philosophy is not to immediately reach a solution but to develop a 
thinking process (problem solving mindset) that leads to the solution. I believe every 
student is different and has a unique style of learning, and my role is to help them find 
and hone their style.


When I started teaching as an assistant professor at the International Institute of Information 
Technology (IIIT), Bhubanesware, India,
my single biggest challenge was keeping the students engaged in my class, especially the first year 
students in C programming course. At the IIIT, I taught C programming to 
first year students, Compiler Design and Java programming to third year students, and
Cryptography to final year (4th year) students. In each course, every single 
problem,  more or less, boiled down to keeping the students engaged in a topic. 
In order to keep them engaged in a class, I took a Coursera course on 
learning \textit{learning-how-to-learn} and 
read many academic articles and non-academic articles about effective learning.
I tried some of the techniques suggested in the Coursera course and 
academic and non-academic articles in my classes. Below I 
describe my experiences with teaching and efforts to engage my students. 





\subsection{Setting Clear Goals}
In every course, I started with the end goal of the course. For example, 
in C programming course, I told my students that by the end of this 
course they would be 
able to write simple C programs, e.g., calculator, 
validating a debit card, and other simple real-world problems. The rationale was 
to show a vision to excite them 
for learning and instill the feeling of empowerment, from
being a consumer to being a developer of software programs.
In addition, in the beginning of every single class 
I would tell my students explicitly the topics we were going to study and 
their importance. For example, when I taught pointers, I explained a 
very high-level idea of an operating system and usage of pointers  
in operating system to access memory locations.
In addition, I told them
Linux is written in C and encouraged them to check the source code.
It helped the class in understanding the importance of 
every individual topic towards the end goal.



\subsection{Start with Why}
One thing that I learnt by teaching for 3 years is that if you want a concept 
to stick in someone's mind, then start with a \textit{why}. For example, when I introduced functions 
in C programming course, I wanted to convey the need of functions. Therefore, I started the class by 
asking them to write a program, using pen and paper, of \textbf{factorial of a number $n$} ($n!$). 
Every one was comfortable with loops, so it was quick. I then asked them to write 
another program: \textbf{$k^{th}$ power of the factorial of $n$} ($(n!)^k$). 
In this assignment, some added an outer loop to cover the factorial computation, 
and some added another loop after the factorial computation (stored the factorial 
computation result and used it in later computation). 
At this point, I asked the class 
how one could write a program of \textbf{factorial of 
the $k^{th}$ power of the factorial of $n$} ($((n!)^k)!$). The class slowly realised 
that they needed to duplicate a lot of code. I then introduced functions 
and showed them the solution of the previous problem by function composition.


\subsection{Catching my Mistakes}
To promote active participation, at the beginning of every class I would announce that 
on a few occasions I would make deliberate mistakes in solving a problem, and the
goal of the class was to catch me on the spot.
If the class succeeded, they received  
class participation marks, otherwise I told them my 
mistake and no one received any marks.
In every class, especially in programming, 
first I would teach a topic, and later I would solve some problems on 
a blackboard, related to the topic. 
It was during the problem solving where I committed deliberate mistakes.
It increased the class participation by an order of 
magnitude. In every single 
feedback that I got during my 3 years of teaching, 
almost everyone appreciated this idea of making deliberate mistakes. 



\subsection{Projects for Learning}
For every course that I taught, I designed projects related to the concepts 
and ideas presented in the course. It was a part of my teaching philosophy to impart 
critical thinking. Furthermore, I asked my students to come up with their own ideas to 
promote idea-exploration, a key step to develop a critical thinking and problem solving mindset. 
In the process of idea-exploration, the students would come up with various interesting 
ideas, but there were many instances when I felt very happy. For example, a group of third year students  
wanted to understand a DNS server by writing their own. Even though they 
could not write a DNS server from scratch successfully, they managed to understand 
most of the workings of a DNS server by downloading a C code from the Internet. 
I consider it a trophy for myself, given that IIIT is a very small technical school. 
In addition, because of my own 
competitive programming background, 
I redesigned the C programming Lab and introduced competitive 
programming, which led to first ACM-ICPC team participation 
from IIIT. More importantly, it made many of the students curious about 
programming, which was considered a difficult subject.

\section{Potential Courses}
I feel qualified to teach most of the computer science courses 
because of my background in computer science, but 
my natural preference is the courses close to my research 
area, e.g., theorem proving, cryptography, logic 
and its applications in computer science, discrete 
mathematics, algorithms and data structures, 
introductory programming course (C/C++/Java/Python/Haskell/OCaml),
foundation of computing, etc. 
In addition, I would also like to design a course to teach 
various voting methods used around the world and 
their advantages and disadvantages from a social choice theory perspective. 
Apart from these topics, I will 
be more than happy to teach other courses, given enough 
preparation time, at introductory and 
intermediate levels, e.g., type theory, theory of programming 
languages, networking, operating systems, databases, etc.


My lab will be a diverse place where students and researchers from formal 
verification, cryptography, 
political science, social choice theory will interact, discuss, 
collaborate on the ideas that matters to democracies and societies.
I look forward to hearing from you. Please let me know if you have any questions. \\
 

\vspace{0.5cm}


\makeletterclosing

\end{document}


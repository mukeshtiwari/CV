\documentclass[a4paper]{article}
\title{Research Statement}
\author{Mukesh Tiwari}
\date{}
\usepackage{url}
\setlength{\topmargin}{-10mm}
\setlength{\textwidth}{7in}
\setlength{\oddsidemargin}{-8mm}
\setlength{\textheight}{9in}
\setlength{\footskip}{1in}


\begin{document}
\fontsize{12}{15}
\selectfont
\maketitle



Fair elections are the only way to keep a democracy alive. In an election, most countries (election commission) use paper ballots to record 
its citizens' votes (preferences). Election commission then produces the final tally from these votes, 
while many scrutineers --workers of the participating political parties and members of general 
public in some countries, e.g., France-- observe the counting process. 
These scrutineers and paper ballots, both, are crucial to ensure that 
the election is free from fraud. Scrutineers keep the counting process transparent by making sure all the vote counting 
rules are followed, while paper ballots ensure verifiability because in case of a dispute these paper ballots 
can be counted again. Many countries, however, in recent years are introducing computers to conduct some part, or all, of 
election processes because it is cost effective, accessible to disabled voters, faster result, convenient, etc. 
In addition, some voting methods are very complex to count by hand. For example,
single transferable vote used in Australian senate elections, and it may take months to declare the final result of 
a senate election if 
counted by hand. Therefore, the Australian Election Commission scans all the ballots of the senate election and 
uses a (closed source) software program to produce the final tally. Moreover, several cantons in Switzerland use an electronic 
voting system developed by SwissPost, India uses electronic voting machines, various municipalities in France 
use electronic voting machines, etc.  Nonetheless, there is a growing debate about using electronic voting machines (computers)
because of software bugs leading to unintended consequences. For example, Rina Mercuri, 
a candidate in Australia, lost an election because of 
a software bug\footnote{\url{https://www.areanews.com.au/story/3971893/mercuri-robbed/}}, 
Swiss post e-voting source code contained a serious 
bug\footnote{\url{https://bit.ly/2TDlQ0j}}, security analysis of Indian electronic 
voting machines have shown many vulnerability \cite{Wolchok:2010:SAI:1866307.1866309}, and 
these few instances are only the tip of the iceberg because many vendors, producing 
e-voting software programs, are not open to criticism\footnote{\url{https://bit.ly/3GYHQuC}}.
Most of 
these software programs are used across the world by various nations to conduct elections for a public office 
lack quality measures and public audit \cite{10.1145/3014812.3014837, 9152765, AEC:2013:LMM}. 
Therefore, it is more imperative than ever  that  we mathematically prove correct (formally verify)  the software programs 
used in elections. It will make the elections more trustworthy and establish the trust of general members of public
in electronic voting.




\section{PhD Work}
My PhD research was focused on verifying electronic voting, specifically vote-counting schemes, in 
the Coq theorem prover. The goal was to 
bring  three important ingredients, correctness, privacy, and (universal) verifiability, of a paper ballot election to 
an electronic setting (electronic voting). In a paper ballot election, correctness and verifiability are
ensured by scrutineers, while privacy 
comes for free because of secret paper ballots.  However, achieving these three desirable properties 
are difficult in electronic voting because software programs used in 
various stages of an (electronic) election work in a opaque (blackbox) manner. This 
opaqueness was one of the major reason for Bundesverfassungsgericht, the German Constitutional Court, 
to rule the usage of electronic voting machines unconstitutional. However, it did not 
completely rule out the usage of electronic voting machines as long as the outcome of an election 
is verifiable, i.e., it is possible for the citizen to check the essential steps in the 
election act and in the ascertainment of the results reliably and without special 
expert knowledge\footnote{\url{https://www.bundesverfassungsgericht.de/SharedDocs/Entscheidungen/EN/2009/03/cs20090303_2bvc000307en.html}} (verifiability).





In my thesis, I demonstrated the correctness of a vote-counting software program 
by implementing and proving the correctness of the Schulze method \cite{10.1007/978-3-319-66107-0_26} in the Coq theorem 
prover. 
The Schulze method is a preferential (ranking) voting method where voters rank the participating 
candidates according to their preferences. It is one of the most popular voting method amongst the open-source projects and 
political groups\footnote{\url{https://en.wikipedia.org/wiki/Schulze_method#Users}}.
While no preferential voting scheme can guarantee all
desirable properties that one would like due to Arrow's impossibility theorem \cite{arrow1950difficulty}, 
the Schulze method offers a good compromise with a number of important properties established by economists, 
social choice theorists, and political scientists. From my Coq implementation of the Schulze method,
I used Coq's extraction mechanism to get an OCaml program.
Then I used OCaml compiler to compile the extracted OCaml program to get an executable  
to count ballots \cite{10.1007/978-3-319-66107-0_26}.
In addition to correctness, my implementation also
ensured (universal) verifiability by producing a scrutiny sheet. The scrutiny sheet contained all 
the data to audit the election independently. The (extracted) OCaml program, however, was 
very slow and could not count more than 10,000 ballots. Therefore, I wrote another 
Coq implementation \cite{bennett2017no}, proven equivalent to the slow Coq 
implementation  \cite{10.1007/978-3-319-66107-0_26}, from which the (extracted) 
OCaml program was able to count millions of ballots.

In both, slow and fast, Coq implementations, I assumed that (preferential) ballots were in plaintext, i.e., 
ranking on every ballot was in a (plaintext) number.  Preferential ballots, 
however, admit ``Italian'' attack \cite{Otten, Benaloh:2009:SSC}. 
If the number of participating candidates are significantly high in 
a preferential ballot election,
then a ballot can be linked to a particular voter if published publicly, on a bulletin board.
The attack is: a coercer demands a voter to mark them as first and for the rest of candidates
in a certain given order (permutation). Later, the coercer checks if that the order appears 
on the bulletin board or not. Although my previous two Coq implementations \cite{10.1007/978-3-319-66107-0_26, bennett2017no} 
satisfied correctness and verifiability criteria, they lacked privacy due to ``Italian'' attack. 
In order to avoid this attack on the Schulze method, I used a homomorphic encryption 
to encrypt the ballots and computed the winner by combining all the 
(encrypted) ballots, without decrypting any individual ballot (privacy). 
Moreover, I addressed verifiability by generating a scrutiny sheet (certificate) 
augmented with zero-knowledge-proofs for various claims, e.g., honest decryption, honest shuffle \cite{10.1007/978-3-030-41600-3_4}. 
This work was carried out in the Coq theorem prover, so I ended up achieving
correctness, privacy, and verifiability. The downside of an encrypted ballot Schulze election, 
or in fact any encrypted ballot election, is difficulty in auditing because of involved 
mathematics of cryptography. Therefore, in the future I want to explore the possibility of a 
verified prototype of scrutiny-sheet checker for encrypted ballots Schulze elections.
Finally, I worked on a verified prototype of a simple approval voting election scrutiny-sheet checker for
International Association of Cryptologic Research (IACR) \cite{10.1145/3319535.3354247}. 
In addition, I was involved in the formalisation of single transferable vote, used in the Australian Senate
\cite{10.1007/978-3-030-00419-4_4}.


\section{Future Work}
My long-term aim is to make formal verification accessible and ubiquitous in 
software development, specifically for the software programs deployed in public domain
that affect common people.
My expertise in \textbf{Theorem Proving, Cryptography, and Election Security}
gives me an unique perspective to solve challenging problems that matter to many democracies 
and its citizens. 

\subsection{Mathematically Proven Correct Cryptographic Algorithms}
	Cryptographic algorithms are used ubiquitously to secure 
	the data and correctness is an utmost requirement for 
	any cryptographic algorithm implementation. Therefore, 
	I will focus on developing mathematically proven correct cryptographic 
	algorithms used in electronic voting, blockchain, and secure communication, e.g., 
  	sigma protocols (zero-knowledge-proof), verifiable (shuffling) mix-networks, 
  	multi-party computations, secret sharing, zk-snark, etc.
	The rationale behind implementing these algorithms is that anyone can use them to construct 
	an utility, e.g., an election scrutiny-sheet checker, 
	a vote-tallying system based on blockchain,
	a verifiable ballot mixing service, an auction server, etc. One of the motivation
	behind this project is to replace the SwissPost Java implementations\footnote{\url{https://bit.ly/3EODmnF}} 
	with mathematically proven correct Coq 
	implementations\footnote{An ongoing project \url{https://github.com/mukeshtiwari/Dlog-zkp/}}.

	

	

\subsection{Mathematically Proven Correct Vote-Counting Algorithms}
In future, I will focus on developing mathematically proven correct
software programs for vote-counting methods used across the world
such as \textit{Single Transferable Vote (STV), 
First Past the Post (FPTP), Instant-runoff voting(IRV), etc.,} in the Coq theorem prover. 
All the vote-counting methods, by design, lend themselves well to computing the 
winner from plaintext ballots; however, so far there is very little research in computing 
the winner from encrypted ballots, while ensuring correctness, privacy, and 
verifiability. Therefore, producing the winner from encrypted 
ballots is a challenging task.
The motivation for this project is that once we have  mathematically proven correct 
components, anyone --election commission or members of general public-- can use them 
to conduct elections, referendums, and verify elections' outcome 
without worrying about software bugs.  The cryptographic algorithms formalised in 
the previous step are going to be used as a building block in this project.



 

\subsection{Mathematically Proven Correct Decentralised Application}
I will focus on a mathematically proven correct decentralised peer-to-peer technical solution 
\cite{liu2004linkable, Clarke2001, schimmer2009peer, 10.1145/1866307.1866346} in the Coq theorem prover. The motivation 
is to help 
whistleblowers in leaking documents and exposing corruption without revealing their identities.
Being vocal against the government is one the most fundamental right of any citizen, but many 
authoritative governments do not appreciate dissent of any form. Therefore, it uses 
its powerful machinery to punish dissidents, in the name of national security. 
The inspiration for this project comes from David McBridge\footnote{\url{https://en.wikipedia.org/wiki/David_McBride_(whistleblower)}} and 
Richard Boyle\footnote{\url{https://bit.ly/3OQ6kbC}}.
David McBride 
is facing a threat of lifetime jail after
leaking the material alleging war crimes by members of the Australia's Special Operations
Task Group in Afghanistan, while Richard Boyle is facing 161 years for exposing the corruption 
inside the Australian Taxation office
(Australia is ranked very high in 
democracy index\footnote{\url{https://worldpopulationreview.com/country-rankings/democracy-countries}}). 
This research will open the door of collaboration with many groups working in verified 
networking, and verified distributed systems. 

\subsection{Mathematically Proven Correct Social Choice Properties}
Computational social choice theory is a research area that is concerned
with aggregation of ballots (preferences)  of multiple voters (agents) and encompasses 
computer science, mathematics, economics, and political science. Typical applications of 
computational social choice theory is voting (preference aggregation), resource allocation, and fair division.
Most of the proofs in computational social choice theory are pen-and-paper proofs, 
and one of my long term future research goal is to make them more precise using the Coq theorem prover.

My current focus is voting because voting methods admit many excellent (social choice) properties
established by political scientists, social choice theorists, 
and economists. For example, the Schulze method follows Condorcet criterion, reversal symmetry,
polynomial runtime, etc., so when we formalise the Schulze method, or in fact any vote-counting method, 
we can push the boundary of correctness by proving that our 
implementation of the Schulze method also follows all the properties \cite{tiwari2021machine}. 
In addition, we can analyse these voting
methods from computational complexity perspective of bribery, if by bribing a certain amount 
voters a specified candidate can be made an election's winner \cite{faliszewski2006complexity}. 
This research opens the door of collaboration with political scientists, 
social choice theorists, economists, and game theorists.



\section{Current Work at Cambridge and Previous Work at Melbourne} 
In my current project \emph{Combinators for Algebraic Structures}\footnote{\url{https://www.cl.cam.ac.uk/~tgg22/CAS/}}, 
I am formalising various graph algorithms on \emph{semiring} algebraic 
structure and combinators (functions) to 
combine two, or more, algebraic structures. In this work, I am developing 
a mathematical correct-by-construction \cite{10.1007/978-3-319-66107-0_26} 
framework, in Coq thereom prover, based on theory of generalised 
path-finding algebra \cite{10.1093/imamat/15.2.161, 10.1145/1080091.1080094}. 
In our framework, depending on concrete instantiation 
of semiring operators, the same algorithm can compute shortest path, longest paths, 
widest paths, multi-objective optimisation, data flow of imperative programs, and many more \cite{gondran2008graphs}. 
In fact, the Schulze method is one instance of our framework. However, 
the current CAS implementation is highly focused towards networking protocols,
so as a future work I will focus on adding more algorithms in CAS related 
to multi-objective optimisation, data flow analysis of imperative programs, and voting. 


At the university of Melbourne, my work was focussed on constant-time implementations, 
a key requirement for many applications including cryptography. 
In particular, I worked on security concurrent separation logic for formally reasoning about 
the information flow properties of a concurrent program. 
I used SecureC, a tool developed at the university of Melbourne, to formalise an email server, 
an auction server, and a location server. 
In addition, I developed a information flow secure gradient descent algorithm (a machine learning 
algorithm) in SecureC for 
trusted execution environment, e.g., Intel SGX and ARM TrustZone to process highly sensitive 
data such as parents' income, race, gender, incarceration time, etc\footnote{https://opportunityinsights.org/}.
This work has been informally presented at PaveTrust workshop\footnote{\url{https://bit.ly/3XJxhBv}}.
All these works were mathematically proven to leak no sensitive 
information to an attacker observing the execution of a program processing 
some secret data.

My lab will be a diverse place where students and researchers from formal 
verification, cryptography, 
political science, social choice theory will interact, discuss, 
collaborate on the ideas that matters to democracies and societies.


\bibliography{research-statement-bibliography} 
\bibliographystyle{unsrt}

\end{document}
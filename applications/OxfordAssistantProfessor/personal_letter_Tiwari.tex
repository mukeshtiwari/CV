\documentclass[11pt,a4paper,roman]{moderncv}      
\usepackage[english]{babel}

\moderncvstyle{classic}                            
\moderncvcolor{black}                            

% character encoding
\usepackage[utf8]{inputenc}
\usepackage{url}

% adjust the page margins
\usepackage[scale=0.90]{geometry}
%\linespread{0.90} 
% personal data
\name{Mukesh Tiwari}{}
\email{mt883@cam.ac.uk}
\phone[mobile]{+447824648138}               
\address{Cambridge, United Kingdom}


\begin{document}

\recipient{To}{The Hiring Committee \\ The University of Oxford, Oxford, UK}
\date{}
\opening{\textbf{Application for the post of Assistant Professor in Cyber Security}}
\closing{Your Sincerly, \vspace{-1em}}



\makelettertitle


Dear Hiring Committee, 
\\
%references such as what and how you got this information
\vspace{1em}
My name is Mukesh Tiwari and I am a senior research associate at 
the University of Cambridge, UK with expertise in  formal methods, cybersecurity and privacy, and social choice theory. 
I am writing to apply for the job \textbf{Assistant Professor in Cyber Security at the University of 
Nottingham}. I have extensive research experience in
formal verification (Coq theorem prover), electronic voting, and cryptography.
My research touches the lives of common people and solves 
real-world problems that matter to democracies. For example, my paper (i)
\textbf{Assume but Verify: Deductive Verification of Leaked Information in Concurrent Applications},
accepted in ACM CCS, develops a theory using the information-flow security principals 
for processing sensitive data --ethnic origin, political opinions,
health-related data, and  biometric data-- of common 
people in secure enclave, e.g., Intel SGX, Arm TrustZone. Moreover, 
we demonstrate the usability of our method by developing non-trivial case studies that handles 
sensitive data accompanied by the machine-checked mathematical proofs that none of them have unintended side-channel 
data leakage; 
(ii) \textbf{Machine-checking Multi-Round Proofs of Shuffle: Terelius-Wikstrom and Bayer-Groth}, 
published in USENIX Security, mathematically establishes a critical piece of 
code in the SwissPost voting software --used in legally binding 
elections in Switzerland-- is correct (and debunks a decade old myth of the cryptographic 
community that Terelius-Wikstrom method is zero-knowledge-proof. We have formally 
proved in the Coq theorem prover that it is a zero-knowledge-argument and not 
a zero-knowledge-proof);
(iii) \textbf{Verifiable Homomorphic Tallying for the Schulze Vote Counting Scheme}, published 
in VSTTE, not only developes a publicly verifiable method to count encrypted ballots for a complex voting method but it 
is also proven correct in the Coq theorem prover to ensure that there is no gap between 
the pen-and-paper proof and the actual implementation;  (iv) \textbf{Verified Verifiers for 
Verifying Elections}, published in ACM CCS, develops a mathematically proven correct tool 
in the Coq theorem prover to verify the elections conducted by 
the International Association for Cryptologic Research. We have used 
our tools to verify the integrity of IACR elections; (v) \textbf{Modular Formalisation and 
Verification of STV Algorithms}, published in E-Vote, develops a mathematically proven 
correct tool in the Coq theorem prover. We have used this tool to verify
the results of Australian Senate election; 
(vi) \textbf{Verifpal: Cryptographic Protocol Analysis for the Real World}, publised in 
INDOCRYPT, develops a tool that can used to model real work cryptographic protocol, and 
Verifpal has been used by many researchers to model security and privacy aspect of 
digital contact tracing during COVID, etc.
At Cambridge, I am developing a mathematically proven correct tool in the Coq theorem prover 
that can be used by networking researchers to model networking-protocols in the abstract 
setting of semirings.
In a nutshell, all my research so far has an impact on the lives on common people and 
researchers.  



I believe communication is the key to resolve any conflict. 
Most of my projects are in a team with one or two other researchers and there are
many situations of conflicting opinions about the ongoing research. However, no matter 
how difficult the situation is, my baseline is to always be respectful to 
the person working with me and then resolve the conflict via dialogue.  
Even though I am an introvert, 
all the researchers that collaborated with me 
enjoyed working with me. In addition, many of my students from India enjoyed working with 
me on their masters project and because of the limitation of two masters project per year, 
I had to say no to many other students.




\vspace{0.5cm}


\makeletterclosing

\end{document}


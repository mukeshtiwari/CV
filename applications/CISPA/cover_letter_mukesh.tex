



%Author by Rajib Das Bhagat (rajibdasbhagat@gmail.com)
%
\documentclass[11pt,a4paper,roman]{moderncv}      
\usepackage[english]{babel}

\moderncvstyle{classic}                            
\moderncvcolor{black}                            

% character encoding
\usepackage[utf8]{inputenc}
\usepackage{url}

% adjust the page margins
\usepackage[scale=0.90]{geometry}

% personal data
\name{Mukesh Tiwari}{}
\email{mt883@cam.ac.uk}
\phone[mobile]{+447824648138}               
\address{Cambridge, United Kingdom}


\begin{document}

\recipient{To}{The Hiring Committee, \\
    CISPA, Saarbrucken}
\date{}
\opening{\textbf{Application for the Faculty Positions in Security, Privacy, and Cryptography}}
\closing{Your Sincerly, \vspace{-1em}}



\makelettertitle

Dear Hiring Committee, 

I am writing to apply
for the \textbf{Tenure-Track Faculty} job in Security, Privacy, and Cryptography. 
I have extensive research experience in
(election) software security, theorem proving (Coq theorem prover), cryptography,  
and social choice theory. 
I have a PhD from the Australian National University, Canberra, Australia,
and currently, I am working as a senior Research Fellow at the University of 
Cambridge. Before moving to Cambridge, I was a 
research fellow at the University of Melbourne, Australia.



The goal of my PhD was to 
bring  three important ingredients, correctness, privacy, and (universal) verifiability, of a 
paper ballot election to an electronic setting (electronic voting). I 
demonstrated: (i) correctness by implementing and proving the correctness of 
a vote-counting algorithm, the Schulze method, in the Coq theorem prover, 
(ii) privacy by using homomorphic encryption to encrypt the ballots and computed
the winner by combining all the (encrypted) ballots, and 
(iii) verifiability by means of various zero-knowledge-proofs.
At CISPA, as a tenure-track faculty, 
I would like to expand my research area into other areas of 
security and formal verification, e.g., domain specific language to reason about 
functional correctness and security properties --from computational complexity perspective-- of 
cryptographic algorithms, anonymous communication, blockchain, zk-snark,  
information flow security, computational complexity of social choice methods, 
etc. My research has been published in Interactive Theorem Proving (ITP), 
Computer and Communications Security (CCS), Electronic Voting (EVote), 
International Conference on Cryptology in India (IndoCrypt),
and some (finished) works have been submitted to ESOP, USENIX, and SIGCOMM. 
All my research work has been formalised in the Coq theorem prover. 

\vspace{0.5cm}
In my current project \emph{Combinators for Algebraic Structures (CAS)},
I am formalising various graph algorithms on \emph{semiring} and combinators (functions) to 
combine two, or more, algebraic structures. In this work, I am developing 
a mathematical correct-by-construction  
framework, in Coq theorem prover, based on theory of generalised path-finding algebra. 
In our framework, depending on concrete instantiation 
of semiring operators, the same algorithm can compute shortest path, longest paths, 
widest paths, multi-objective optimisation, data flow of imperative programs, and many more. 
In fact, the Schulze method is one instance of our framework. However, 
the current CAS implementation is highly focused towards networking protocols,
so as a future work I will focus on adding more algorithms in CAS related 
to multi-objective optimisation, data flow analysis of imperative programs, and voting. 


As a research associate at the University of Melbourne, I have spearheaded three projects:
 (i) A formally verified auction server, (ii) A formally
 verified location server, and (iii) A formally verified machine learning 
 algorithm that is resistant to side-channel attacks and can  
 run inside the Intel SGX (Software Guard Extensions) enclave for learning
 on sensitive data. All three implementations have been proven 
 memory safe (using separation logic) and free from information
 leaks using the SecureC tool, a tool developed at the University of Melbourne.

I look forward to hearing from you. Please let me know if you have any questions. \\
 

\vspace{0.5cm}


\makeletterclosing

\end{document}












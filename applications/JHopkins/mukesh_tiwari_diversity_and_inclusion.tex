\documentclass[a4paper]{article}
\title{Diversity, Equity and Inclusion}
\author{Mukesh Tiwari}
\date{}
\usepackage{url}
\setlength{\topmargin}{-10mm}
\setlength{\textwidth}{7in}
\setlength{\oddsidemargin}{-8mm}
\setlength{\textheight}{9in}
\setlength{\footskip}{1in}
 
\begin{document}
\fontsize{12}{15}
\selectfont
\maketitle

Diversity and inclusion matters a lot to me because of my own background (born in 
a farming community in a remote Indian village with no school). I believe everyone is capable of 
achieving everything, all they need is resource and support. Therefore,  I firmly believe that 
all institutions in a democracy, including universities should strive to build
community of individuals with diverse backgrounds and life experiences, 
free of discrimination based on racial and ethnic origin, gender identity, 
sexual orientation, social economic status or religious belief. 

\vspace{0.5cm}
In the past, when I was working as an assistant professor in International Institute of 
Information Technology (IIIT) Bhubanesware, I participated in teaching 
workshops, organised by the student body of IIIT, to teach the 
children of migrant workers, working as a construction worker in IIIT.
In addition, every year when I visit my village 
in summer, I teach physics, mathematics, and basic computer literacy course to 
all the children, coming from 
economically underdeveloped community, working in 
agricultural field in evenings. However, I lack funds to buy computers for everyone 
and therefore, I have been recently trying to encouraging people to donate 
their computers. My long term goal is to lower the gap of digital divide and make the 
children from economically underdeveloped community literate in computer education 
and employable in computer industry to break the cycle of poverty. 
I am committed to promoting diversity, equity, and inclusion (DEI) in my lab by 
hiring students from diverse backgrounds. I also intend to invite 
students from underdeveloped and developing countries to the University of 
British Columbia to given them exposure to a world class research facilities. 



\vspace{0.5cm}
I am also a staunch supporter of a robust and transparent 
democracy, and one of my future goal is to end the 
first-past-the-post (FPTP) voting method in India by challenging the Election 
Commission of India in the Indian Supreme Court. FPTP is an unsatisfactory method 
to elect the winner of an election and it is actively exploited by 
the political parties in India (sometimes, a party with even 33\% votes 
wins the state election). Therefore, elections in India is no longer 
about the voters but about strategic manipulation. Consequently, 
in recent elections, many voters are boycotting the elections and 
not casting their ballots. 






\end{document}
  \documentclass[a4paper]{article}
\title{Proposal}
\author{Mukesh Tiwari}
\date{}
\usepackage{url}
\setlength{\topmargin}{-10mm}
\setlength{\textwidth}{7in}
\setlength{\oddsidemargin}{-8mm}
\setlength{\textheight}{9in}
\setlength{\footskip}{1in}



\begin{document}
\fontsize{11}{15}
\selectfont
\maketitle




\section{Objective:} 

In this project, our goal is to develop a mathematically proven correct 
and secure software (computer program) used in Switzerland's  electronic (internet)
voting, employed for elections  and referendums, using formal methods.

\section{Background and Problem:}

Switzerland is one of the oldest democracy, and its citizens 
actively participate in regular decision making, four times in a year,
and express their opinion by means of voting. 
In general, paper ballots are used by Swiss citizens to express their 
opinion; however, in recent years there is a growing debate 
in favour of electronic voting to address various problems, such as including more 
of its overseas voters, being more inclusive 
to towards visually impaired voters\footnote{\url{https://www.usenix.org/conference/enigma2021/presentation/folini}}, 
minimising the number of informal paper ballots, and streamlining the vote counting 
process \cite{7114482, DBLP:conf/ev/Driza-MaurerSTW12}.
In fact, electronic voting is already in practice since 2000 in 
Switzerland, and the first trial of electronic voting to 
a binding referendum was conducted in Geneva in 2003, followed 
by Neuch\^{a}tel and Zurich  in 2005  \cite{gerlach2009three}.
Geneva developed its own software for electronic voting, 
meanwhile Zurich took the help of a private company Unisys \cite{4454399}
and Neuch\^{a}tel developed with the help of their own IT department in collaboration
with another private company Scytl \cite{reiners2020vote}. 
Other cantons in Switzerland  
used the software developed by Geneva, Zurich, and Neuch\^{a}tel to conduct their elections and referendums.
Nonetheless, currently none of the cantons 
are developing their own software citing 
cost\footnote{\url{https://www.swissinfo.ch/eng/politics/digital-voting_geneva-shelves-e-voting-platform-on-cost-grounds/44577490}}
and security, and they are  
using the software develop by SwissPost, with the help of Scytl, 
who entered into the competition in 2016 \cite{10.1007/978-3-031-15911-4_4}.
SwissPost and Scytl kept the source code of their software closed until 2019, but in 2019 
they hosted it on GitLab\footnote{\url{https://gitlab.com/swisspost-evoting/}}.
It only took few months for 
researchers to break its cryptographic code, the most important part of 
their software \cite{9152765, locher2019analysis}.
Consequently, Swiss Federal Chancellery stopped the cantons from using 
SwissPost software for 2020 elections and formed a committee of 
experts from various areas including electronic-voting, cryptography, computer network, 
and political science to 
analyse the current voting situation and asked for their 
recommendations\footnote{\url{https://www.bk.admin.ch/bk/en/home/politische-rechte/e-voting/berichte-und-studien.html}
 (\textit{List of mandated experts.pdf})}. 
The participating experts have produced 
their findings and one of the crucial recommendation was to develop the source code 
of an electronic voting software using formal method (mathematically 
prove correct code)\footnote{\url{https://www.bk.admin.ch/bk/en/home/politische-rechte/e-voting/berichte-und-studien.html} (see 
\textit{Summary of the Expert Dialog 2020.pdf})}. Thus, our proposal for a formally verified 
software (computer program) for electronic voting in Switzerland. 


\section{Methodology and Solution}
It is well-understood in computer science community that software have bugs, 
and the most widely used method to eliminate bugs from a software is testing. 
Unfortunately, testing cannot eliminate all the bugs, and therefore, 
mathematicians and computer scientists have put many decades of research to develop a bug-free 
software using formal methods. In this project, we are going to use 
formal methods to develop a mathematically proven correct software
using the Coq theorem prover \cite{the_coq_development_team_2019_3476303}, EasyCrypt \cite{10.1007/978-3-642-22792-9_5}, and 
Jasmin \cite{10.1145/3133956.3134078}. 


In the first phase of this project, our goal is to understand the specification of 
crypto-primitives\footnote{\url{https://gitlab.com/swisspost-evoting/crypto-primitives/crypto-primitives/-/blob/master/Crypto-Primitives-Specification.pdf}}, 
and implement  and prove the crypto-primitives correct in the Coq, EasyCrypt, and Jasmin theorem prover. From our implementation, 
we intend to extract a mathematically proven correct software for the frontend 
that runs in the browser interacting with a voter and the backend that runs on the server of an election. 
Our goal is to extract both, frontend and backend, from the same formalisation. Currently, 
SwissPost has written the same algorithms in two languages: (i) Java for the backend, and 
(ii) TypeScript for the frontend.  In principle, writing the same algorithm in 
two different languages may appear innocuous, but in 
practice it is problematic because they are, in general, written by two 
different teams and there is a possibility of two code bases diverging from each other on some corner cases, 
i.e., outputting two different values for the same input. On the other hand, 
we are using just one formalism to extract two code bases, divergence is 
not applicable for our case. 
The Coq theorem prover has code generator that produces web assembly code (frontend) from 
a Coq formalisation, but it does not produce a provable correct assembly code (backend). 
On the other hand, EasyCrypt and Jasmin  produce a provable correct assembly code (backend) from 
its formalisation, but it lacks web assembly code (frontend) and
therefore extending it to extract web assembly code will also be a part of this phase.  
We expect the duration of this phase to be 1 year. 



In the second phase of this project, our goal is to understand the specification of the election 
verifier\footnote{\url{https://gitlab.com/swisspost-evoting/e-voting/e-voting-documentation/-/blob/master/System/Verifier_Specification.pdf}}
and implement it in the Coq, EasyCrypt, and Jasmin theorem prover. From our implementation, 
we intend to extract a mathematically proven correct software that takes the cryptographic 
evidence of an elections and certifies if every step has been performed correctly or not. The rationale 
to produce another verifier is two fold: 
\begin{itemize}
  \item the current verifier\footnote{\url{https://gitlab.com/swisspost-evoting/verifier/}} 
is written in Java and the only way to 
find a bug in Java program is to write test cases. However, as we eluded above, it is not 
possible to catch every single bug by means of testing, and therefore, it is possible that 
Java verifier may return incorrect result for the cryptographic evidence of an elections. On the 
other hand, our verifier is mathematically proven correct and hence it will never return incorrect result.
In addition, our verifier can be used to detect bugs in Java verifier by feeding a random input 
to our verifier and comparing the output of our verifier to the Java verifier's output for the 
same random input. 

\item having an independent verifier will boost the confidence of voters in the system and 
reduce the barrier of electronic voting acceptance because the verifier will facilitate 
an additional independent way to verify an election's outcome \cite{rivest2008notion}. 
Our verifier can be used by voters, election scrutineers, political parties to ensure the 
integrity of an election. 
Cryptographic primitives developed in the first phase will be used in this phase. 
We expect the duration of this (second) phase to be 1 year.

\end{itemize}

In the third phase of this project, our goal is to understand the whole 
e-voting system architecture\footnote{\url{https://gitlab.com/swisspost-evoting/e-voting/e-voting-documentation}} 
and how various components, e.g., voting client, voting server, return codes, etc., 
interact with each other. We would like to model this 
interaction mathematically using session types \cite{10.1007/978-3-642-38592-6_5}. 
Session types are a formalism for distributed communicating processes, i.e., 
they encode various piece of information in a type system to mathematically reason about 
distributed communicating processes. It has found applications in numerous domains, e.g, 
distributed computing \cite{10.1145/3290341}, 
multi-threaded functional programming \cite{vasconcelos2006type},
component-based systems \cite{vallecillo2003typing}, web programming \cite{10.1145/3446804.3446854}, etc. 
However, Coq and EasyCrypt, in their current form, are not capable of reasoning about session types. 
Therefore, our first step will be extend these tools to reason about 
session types \cite{ZFHNY2020} and then use them to reason about the interaction between 
various components. This phase will be in two parts:
\begin{itemize}
  \item the first part will be to develop a formal abstract model of interaction.  Given that session types 
  make the structure of communication explicit, we will get a formal model of interaction for 
  the e-voting system architecture that can be used to reason about 
  various properties, e.g., communication safety \cite{TONINHO201761},
  secure information flow \cite{10.1007/978-3-642-15375-4_17}, computational complexity \cite{10.1145/3209108.3209122},
  etc. 

  \item the second, and final, part will be to develop an implementation that
  respects the abstract model, produced in the first part. In addition, 
  we can also check if the current Java implementation follows our 
  abstract model or not, which will be a valuable insight about the current Java implementation.
  In this way, we close all the gaps by producing end-to-end verifiable system \cite{4271648, ZFHNY2020, basin}.
  We expect the duration of the third phase to be 1 year.

\end{itemize}







  
\section{Deliverables:} 

This project, if funded, will produce the following:
\begin{itemize}
  \item output of the first phase: a mathematically proven correct software for the frontend and backend 
    for cryptographic primitives (project duration: 1 year). 
  \item output of the second phase: a mathematically proven correct software that takes the cryptographic evidence of an elections
    and certifies if every step has been performed correctly or not (project duration: 1 year). 
  \item output of the third phase: a mathematical model of interaction and its implementation 
    for the e-voting system architecture (project duration: 1 year).
  
\end{itemize}

Our mathematically proven correct software written in Coq, EasyCrypt, and Jasmin 
can be used to test the current Java code written by SwissPost. In addition, we 
will significantly improve the understanding of interaction by producing a formal model 
in the third phase of this project. This project, if funded, will 
not only produce the mathematically proven correct software but it will also 
enhance our understanding the whole system that include interaction amongst 
various components and the security of voting protocol. 

In addition, our project will bring more transparency because not only we will 
open source the software but we will also publish papers in security conferences. 
Our claims will go through a rigorous review process of the program committee --leading 
security experts-- of security conferences, 
a very different situation from software companies' practices. Companies providing software for 
electronic voting always cite intellectual property to not release the 
source and create a very opaque situation. This opaqueness was one of the major reason for
Bundesverfassungsgericht\footnote{\url{https://www.bundesverfassungsgericht.de/SharedDocs/Entscheidungen/EN/2009/03/cs20090303_2bvc000307en.html}}, 
the German Constitutional Court, to rule the usage of electronic voting
machines unconstitutional. It, however, did not completely rule out the usage of electronic voting
machines as long as the outcome of an election is verifiable, i.e., it is possible for citizens to check
the essential steps and ascertain the results reliably and without
any special expert knowledge (verifiability).



\bibliography{reference}
\bibliographystyle{unsrt}



\end{document}
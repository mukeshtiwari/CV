\documentclass[a4paper]{article}
\title{Teaching Statement}
\author{Mukesh Tiwari}
\date{}
\usepackage{url}
\setlength{\topmargin}{-10mm}
\setlength{\textwidth}{7in}
\setlength{\oddsidemargin}{-8mm}
\setlength{\textheight}{9in}
\setlength{\footskip}{1in}
 
\begin{document}
\fontsize{12}{15}
\selectfont
\maketitle


My teaching philosophy is to not immediately reach a solution but to develop a 
thinking process (problem solving mindset) that leads to the solution. I believe every 
student is different and has a unique style of learning, and my role is to help them find 
and hone their style.

\section{Teaching Experience}
When I started teaching as an assistant professor at the International Institute of Information 
Technology (IIIT), Bhubanesware, India\footnote{It is a small technical school},
my single biggest challenge was keeping the students engaged in my class, especially the first year 
students in C programming course. At the IIIT, I taught C programming to 
first year students, Compiler Design and Java programming to third year students, and
Cryptography to final year (4th year) students. In each course, every single 
problem,  more or less, boiled down to keeping the students engaged in a topic. 
In order to keep them engaged in a class, I took a Coursera course on 
learning\footnote{\url{https://www.coursera.org/learn/learning-how-to-learn}} and 
read many academic articles and non-academic articles about effective learning.
I tried some of the techniques suggested in the Coursera course and 
academic and non-academic articles in my classes. Below I 
describe my experiences with teaching and efforts to engage my students. 





\subsection{Setting Clear Goals}
In every course, I started with the end goal of the course. For example, 
in C programming course, I told my students that by the end of this 
course they should be 
able to write simple C programs, e.g., calculator, time tracking system,
student attendance system that automatically calculates their attendance percentage, 
validating a debit card, and other simple real-world problems. The rationale was 
to show a vision to excite them 
for learning and instill the feeling of empowerment, by a narrative from
being a consumer to being a developer of software programs.
In addition, in the beginning of every single class 
I would tell my students explicitly the topics we were going to study and 
their importance. For example, when I taught pointers, I explained a 
very high-level idea of an operating system and usage of pointers  
in operating system to access memory locations.
In addition, I told them
Linux is written in C and encouraged them to check the source code.
It helped the class in understanding the importance of 
every individual topic towards the end goal.



\subsection{Start with Why}
One thing that I learnt by teaching for 3 years is that if you want a concept 
to stick in someone's mind, then start with a \textit{why}\footnote{I learnt this idea by reading the book 
\textit{Start With Why} by Simon Sinek.}. For example, when I introduced functions 
in C programming course, I wanted to convey the need of functions. Therefore, I started the class by 
asking them to write a code, using pen and paper, to compute the \textbf{factorial of a number $n$}. 
Every one was comfortable with loops, so it was quick. I then asked them to write 
another program that computes the \textbf{$k^{th}$ power of the factorial of $n$}. 
In this assignment, some added an outer loop to cover the factorial computation, 
and some added another loop after the factorial computation (stored the factorial 
computation result and used it in later computation). 
At this point, I asked the class 
how one could write a program that computes the \textbf{factorial of 
the $k^{th}$ power of the factorial of $n$}. The class slowly realised 
that they needed to duplicate a lot of code. I then introduced functions 
and showed them the solution of the previous problem by function composition.


\subsection{Catching my Mistakes}
To promote active participation, at the beginning of every class I would announce that 
on a few occasions I would make deliberate mistakes in solving a problem, and the
goal of the class was to catch me on the spot.
If the class succeeded, they received  
class participation marks, otherwise I told them my 
mistake and no one received any marks.
In every class, especially in programming, 
first I would teach a topic and later solved some problems on 
a blackboard, related to the topic. 
It was during the problem solving where I committed deliberate mistakes.
It increased the class participation by an order of 
magnitude\footnote{I learnt this 
from advertising agencies where they write some ads wrong 
intentionally to grab the attention.}. In every single 
feedback that I got during my 3 years of teaching, 
almost everyone appreciated this idea of making deliberate mistakes. 



\subsection{Projects for Learning}
For every course that I taught, I designed projects related to the concepts 
and ideas present in the course. It was a part of my teaching philosophy to impart 
critical thinking. Furthermore, I asked my students to come up with their own ideas to 
promote idea-exploration, a key step to develop a critical thinking and problem solving mindset. 
In the process of idea-exploration, the students would come up with various interesting 
ideas, but there were many instances when I felt very happy. For example, a group of third year students  
wanted to understand a DNS server by writing their own. Even though they 
could not write a DNS server from scratch successfully, they managed to understand 
most of the workings of a DNS server by downloading a C code from the Internet. 
I consider it a trophy for myself, given that IIIT is a very small technical school. 
In addition, because of my own 
competitive programming background, 
I redesigned the C programming Lab and introduced competitive 
programming\footnote{\url{https://www.topcoder.com/community/arena}, 
\url{https://www.spoj.com/}}, which led to first ACM-ICPC team participation 
from IIIT. More importantly, it made many of the students curious about 
programming, which was considered a difficult subject.

\section{Potential Courses}
I feel qualified to teach most of the computer science courses 
because of my background in computer science, but 
my natural preference is the courses close to my research 
area, e.g., theorem proving, cryptography, logic 
and its applications in computer science, discrete 
mathematics, algorithms and data structures, 
introductory programming course (C/C++/Java/Python/Haskell/OCaml),
foundation of computing, etc. 
In addition, I would also like to design a course to teach 
various voting methods used around the world and 
their pros and cons from a social choice theory perspective. 
Apart from these topics, I will 
be more than happy to teach other courses, given enough 
preparation time, at introductory and 
intermediate levels, e.g., type theory, theory of programming 
languages, networking, operating systems, databases, etc.



\end{document}
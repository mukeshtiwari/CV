\documentclass[11pt,a4paper,roman]{moderncv}      
\usepackage[english]{babel}

\moderncvstyle{classic}                            
\moderncvcolor{black}                            

% character encoding
\usepackage[utf8]{inputenc}
\usepackage{url}

% adjust the page margins
\usepackage[scale=0.90]{geometry}
%\linespread{0.90} 
% personal data
\name{Mukesh Tiwari}{}
\email{mt883@cam.ac.uk}
\phone[mobile]{+447824648138}               
\address{Cambridge, United Kingdom}


\begin{document}

\recipient{To}{The Hiring Committee \\ King's College, London, UK}
\date{}
\opening{\textbf{Application for the post of Lecture in Computer Science (Software Engineering)}}
\closing{Your Sincerly, \vspace{-1em}}



\makelettertitle


Dear Hiring Committee, 
\\
%references such as what and how you got this information
\vspace{1em}
My name is Mukesh Tiwari and I am a senior research associate at 
the University of Cambridge, UK. I am writing to apply
for the job \textbf{Lecture in Computer Science (Software Engineering)}. 
I have extensive research experience in
formal verification (Coq theorem prover), Cryptography, and Electronic Voting.
My research touches the lives of common people and solves 
many real world problems that matter to democracies. For example, my paper 
\textbf{Machine checking Multi-Round Proofs of Shuffle: Terelius-Wikstrom and Bayer-Groth}, 
published in USENIX Security, mathematically establishes a critical piece of 
code in the SwissPost voting software --used in legally binding 
elections in Switzerland-- is correct (and debunks a decade old myth of the cryptographic 
community that Terelius-Wikstrom method is zero-knowledge-proof. We have formally 
proved in the Coq theorem prover that it is a zero-knowledge-argument and not 
a zero-knowledge-proof);
\textbf{Verifiable Homomorphic Tallying for the Schulze Vote Counting Scheme}, published 
in VSTTE, not only developes a publicly verifiable method to count encrypted ballots for a complex voting method but it 
is also proven correct in the Coq theorem prover to ensure that there is no gap in 
the pen-and-paper proof; 
\textbf{Verified Verifiers for 
Verifying Elections}, published in ACM CCS, develops a mathematically proven correct software 
in the Coq theorem prover to verify the elections conducted by 
the International Association for Cryptologic Research. We have used 
our software to verify the integrity of IACR elections; \textbf{Modular Formalisation and 
Verification of STV Algorithms}, published in E-Vote, develops a mathematically proven 
correct software for single transferable vote algorithm in the Coq theorem prover. We have used this software to verify
the results of Australian Senate election; 
\textbf{Verifpal: Cryptographic Protocol Analysis for the Real World}, publised in 
INDOCRYPT, develops a software that can used to model real work cryptographic protocol, and 
Verifpal has been used by many researchers to model security and privacy aspect of 
digital contact tracing during COVID, etc.
At Cambridge, I am developing a mathematically proven correct software in the Coq theorem prover 
that can used by networking researchers to model networking-protocols in the abstract 
setting of semirings.


In a nutshell, every single project that I have worked so far has produced a mathematically 
proven correct software and they have a far-reaching impact on the lives on common people and 
researchers. I find that the 
King's College London will be a perfect place to continue my real-world impact 
research, given that it is already a leading research university in the UK. 




\vspace{0.5cm}


\makeletterclosing

\end{document}


\documentclass[a4paper]{article}
\title{Research Statement}
\author{Mukesh Tiwari}
\date{\today}
\usepackage{url}
\setlength{\topmargin}{-10mm}
\setlength{\textwidth}{7in}
\setlength{\oddsidemargin}{-8mm}
\setlength{\textheight}{9in}
\setlength{\footskip}{1in}

\begin{document}
\fontsize{12}{15}
\selectfont
\maketitle



My research focuses on formal verification of software programs used in  democratic process, e.g. elections.  
The rationale for verifying these software programs is that various 
critical decisions are taken by government based on the output produced by these software programs. However, if the 
software program involved in the decision making process is bug ridden, then it may produce a wrong 
output which could lead to a poor decision making. As a result, these kind of poor 
decision making could hamper the trust of  members of 
general public in system despite the fact that there were no such intentions by government. 
Worse yet, now a days there are more and more automation of decision making based on machine learning algorithm, 
artificial intelligence, etc.  It is more imperative than ever  that  we formally verify these software programs and 
develop software tools to automate the formal verification process make the decision making more trustworthy. 


\section{PhD Work}
During my PhD, my research was focussed on verifying electronic voting, specifically vote counting schemes, 
and brining three important ingredients, correctness, 
privacy, and verifiability, of paper ballot election to electronic setting (electronic voting). In a paper ballot election, correctness and verifiability is
assured by public scrutineers, appointed by participating candidates and political parties, observing the counting process conducted by 
election commission officials. Moreover, the privacy in paper ballot election comes for free because of the secret ballot mechanism, introduced 
by Australia in 1855, where the voters choices are anonymous. However, none of 
these desirable properties can be achieved in electronic voting because the software programs, used during the 
various stages of  elections, works in very opaque way \cite{Wolchok:2010:SAI:1866307.1866309, 10.1007/978-3-319-22270-7_3}.
 This opacity can cause a harm of various level, but 
it would be devastating if the vote-counting 
software program produces a wrong winner, which is not intended by the voters, because of a software bug. 
In this setting, it is very difficult to 
establish correctness and verifiability because there is no (direct) involvement of human in the counting process (other 
that pressing some buttons to run the software program).  
In most cases, the vote-counting software programs lack the quality measures which could lead to various 
problems including, but not limited to,  producing incorrect result, ballot identification, etc.  More importantly, these software programs
are treated as commercial in confidence and are not allowed to be inspected by members of general public \cite{AEC:2013:LMM}. 
As a consequence, the result produced by these software programs can not be substantiated, which is very detrimental 
for any democratic society. Because of these reasons, the early adopter of electronic 
voting system, Germany and The Netherlands, were the early abandoner \cite{Jacobs2009}. 

In my thesis, I addressed the concern of correctness of vote counting 
software by implementing  and proving the correctness of the implementation (Schulze Method) in Coq theorem prover. 
I also proved that my implementation of Schulze Method follows the various meta properties, e.g. condercet winner, reversal symmetry.
I addressed the privacy concern by using homomorphic encryption to count the (encrypted) ballot without decrypting 
any individual ballot and  verifiability concern by generating a independently checkable scrutiny sheet (certificate) 
augmented with zero-knowledge-proofs for various claims made during the counting \cite{10.1007/978-3-030-41600-3_4, 10.1145/3319535.3354247}. 


\section{Current Work}
Currently I am working as a Post Doctoral researcher with Toby Murray at the university of Melbourne, and I work on 
security concurrent separation logic for formally reasoning the information flow in concurrent programs. My current project 
focuses on a verified email server,  inspired by Hagrid key-server, which leaks no sensitive information to attackers.
The basic idea behind this project is that given all the APIs of a email server, an attacker would not be able to get 
any information which is classified sensitive (high).  We have already presented our work in \textit{VerifyThis 2020} challenge. 


\section{Future Work}
Throughout my past and current research, I have extensively used the formal verification techniques to verify the software programs, 
and I would like to continue to do so in the future as well. In future, I would like to:
\begin{itemize}

\item focus on verifying machine learning algorithm. Now a days, machine learning algorithms are deployed 	
in various domains and decisions are taken based on the output of these programs. In general, 
the most popular language to implement these these algorithms are Python which has 
no concept of type safety. As a consequence, a programmer writing these implementation can make 
a subtle error, which could be catastrophic.  In contrast, a type safe language, e.g. Haskell, 
would catch many of these errors at the compile time. However, none of these languages would 
give 100\% guarantee that the implementation is correct. The only way we can get 100\% guarantee  
by implementation these algorithm in a theorem pover, Agda, Coq, Lean,  and proving that 
the implementation follows the specification.  

\item focus on pushing privacy preserving machine learning algorithm in public domain. 
Currently, all the deployed machine learning algorithms are trained on plaintext data. 
However, it poses a significant challenge if the training data contains  sensitive information.
A privacy preserving machine learning would have all the benefits of machine learning without 
any concern about leaking sensitive data to anyone. Right now, the field is in very nascent phase, 
but given that there is a lot of progress in homomorphic encryption since 2009, it would be 
an excellent idea to explore privacy preserving machine learning. The other possible direction could be
multi-party computation, but these concepts are very 
difficult for machine learning researcher to understand.  Developing a domain specific language which 
abstracts the underlying details would help machine learning researchers to push the boundaries of 
privacy preserving machine learning. 

\item focus on designing probabilistic programming language, either embedded in a theorem prover, or 
 a separate language, for 
	cryptographic implementation with main focus on correctness. Cryptographic algorithms 
	play a critical role in any application focussed on privacy. 
	If any part of implementation is not done right, it could lead to severe consequences 
	(possible identification of activist, voter). The long term (and ambitous) goal would be
	developing a verified compiler which would generate assembly code (having various properties such as 
	memory safe, constant time). There are already some 
	work in this area, e.g. Jasmin and CakeML; however,  Jasmin's goal is 
	to generate formally verified efficient assembly code and CakeML is a general purpose 
	language. None of these are focussed on probabilistic programming which is a 
	necessary component to model any cryptographic algorithm.



\item focus on designing electronic voting protocol and proving the meta properties. Many countries, e.g. Switzerland,
often conduct referendums via internet-voting on various issues that matter to their democracy, and 
it encourages their citizen to participate actively in decision making. It would be 
interesting to design and implement electronic voting protocols for low-coercion 
situations, which would  encourage the voters to actively participate in 
various decision making process without any fear of vote selling. 

\item focus on information flow security in software design (continue my collaboration with Toby Murray). 
Proving formally the information flow between various components of a software program can leads to 
security focussed design. In certain cases, where security is paramount,  
it helps in understanding the potential information leak, which can further be used 
for assessing the repercussions. 


\end{itemize}


Finally, My long-term aim is to make formal verification accessible and ubiquitous in software development, specifically deployed in public domain. 
I believe that my expertise in \textit{Theorem Proving, Election Security, and Cryptography} gives me an unique perspective to solve challenging problem, which 
matters to many democracies. 
\bibliography{research-statement-bibliography} 
\bibliographystyle{unsrt}

\end{document}
\documentclass[a4paper]{article}
\title{Research Statement}
\author{Mukesh Tiwari}
\date{\today}
\usepackage{url}
\setlength{\topmargin}{-10mm}
\setlength{\textwidth}{7in}
\setlength{\oddsidemargin}{-8mm}
\setlength{\textheight}{9in}
\setlength{\footskip}{1in}

\begin{document}
\fontsize{12}{15}
\selectfont
\maketitle

My work aims to build \textbf{correct software programs used in democratic processes}, 
using the Coq theorem prover. 
I focus on formal verification of software programs used in elections, 
cryptographic primitives used in elections, etc.  Various 
critical decisions, e.g., electing the winner of an election, 
are taken by government based on the output produced by these software programs, but if these 
software programs are bug-ridden, then they may produce a wrong 
output. As a result, it could hamper the trust of  
members of general public in the decision making of government. 
Worse yet, now a days there are more and more automation of decision making 
in various bodies of government (democracies) based on software programs, 
and therefore, it is more imperative than ever  that  we formally verify these software programs and 
develop software tools to automate the formal verification process to make the decision making more trustworthy. 

Fair elections are the only way to keep a democracy alive. Most countries (election commission) use paper ballots to record 
its citizens' vote (preference), and then election commission officials produce the final tally from these votes, 
while many scrutineers --workers of the participating political parties and members of general public in 
some countries, e.g., France--  
observe the counting process. These scrutineers and paper ballots, both, are very crucial to ensure that 
the election is free from any fraud. Scrutineers keep the counting process transparent by making sure all the vote counting 
rules are followed, while paper ballots ensure verifiability because in case of a dispute, these paper ballots 
can be counted again. Many countries, however, in recent years are introducing computers to conduct some part, or all, of 
election processes because it is cost effective, accessible to disables voters, faster result, convenient, etc. 
In addition, some voting methods are very complex to count by hand, e.g., 
single transferable vote, used in Australian senate elections, and it may take months to declare the final result if 
counted by hand. Therefore, the Australian election commission scans all the ballots of senate election and 
uses an software program to produce the final tally. In addition, few cantons in Switzerland uses electronic 
voting system developed by Swiss Post, India uses electronic voting machines, various municipalities in France 
uses electronic voting machines, etc.

\section{PhD Work}
During my PhD, my research was focused on verifying electronic voting, specifically vote-counting schemes, in 
the Coq theorem prover
and bringing  three important ingredients, correctness, privacy, and (universal) verifiability, of paper ballot election to 
electronic setting (electronic voting). In a paper ballot election, correctness and verifiability are
assured by scrutineers. Moreover, privacy in a paper ballot election comes for free because of the secret 
ballot, introduced by Australia in 1855.  Achieving these, however, desirable properties --correctness, 
privacy, and verifiability-- can be difficult in 
electronic voting because software programs, used in 
various stages of an election, work in a very opaque (blackbox) manner. In this (electronic) setting, 
a voter cast their ballot in an election and the software programs produces the result without the involvement 
of any human (scrutineers) in the counting process, other than pressing some buttons to run the software program,
when the election finishes. This opacity can cause a harm of various level, including electing a 
wrong winner that is not intended by the voters but because 
of software bugs\footnote{A software bug elected a wrong winner: \url{https://www.areanews.com.au/story/3971893/mercuri-robbed/}}. Most of 
these software programs, used across the world by various nations to conduct elections for a public office, 
lacks quality measures \cite{10.1145/3014812.3014837, 9152765} but more importantly they are treated as commercial in confidence and therefore,
members of general public are not allowed to inspect these artefacts \cite{AEC:2013:LMM}.





In my thesis, I addressed the concern of correctness of vote counting 
software by proving the correctness of the Schulze Method's implementation in 
Coq theorem prover \cite{10.1007/978-3-319-66107-0_26}
(Schulze method is a preferential (ranking) voting method where voters rank the 
candidates according to their preferences. It is one of the most popular voting method amongst the open-source projects and 
political groups\footnote{\url{https://en.wikipedia.org/wiki/Schulze_method#Users}}.
While no preferential voting scheme can guarantee all
desirable properties that one would like to impose due to Arrow's impossibility theorem \cite{arrow1950difficulty}, 
the Schulze method offers a good compromise, with a number of important properties establish by economists, 
social choice theorists, political scientists, etc.) In addition, I also ensured that when my implementation
produces a winner, it also produces a scrutiny sheet that can be checked independently by 
scrutineers (universal verifiability).  However, for this formalisation the extracted OCaml code was 
very slow, so I wrote another (fast) implementation 
that I proved equivalent to the slow one. This implementation was able to count millions of ballots \cite{bennett2017no}.
In both of these formalisation, I assumed that (preferential) ballots were in plaintext, i.e., ranking on 
every ballot was in a (plaintext) number.  Preferential ballots, 
however, admit ``Italian'' attack \cite{Otten, Benaloh:2009:SSC}. 
If the number of participating candidates are significantly higher in 
a preferential ballot election,
then a ballot can be linked to a particular voter if published on bulletin board
(a coercer demands a voter to mark them as first and for the rest of candidates
in a certain permutation. Later, the coercer checks if that permutation appears 
on the bulletin board or not). In order to
avoid this attack on Schulze method, I used homomorphic encryption to count the (encrypted) ballots, without decrypting 
any individual ballot, and addressed verifiability by generating a independently checkable scrutiny sheet (certificate) 
augmented with zero-knowledge-proofs for various claims, e.g., honest decryption, honest shuffle, etc.,  
made during the counting \cite{10.1007/978-3-030-41600-3_4}. Finally, I wanted to develop to formally 
verified scrutiny sheet checker for encrypted ballots Schulze election, but due to lack of 
time\footnote{PhD duration is 3.5 years in Australia} I worked on a scrutiny sheet checker for a simple approval voting election,
International Association of Cryptologic Research (IACR) election \cite{10.1145/3319535.3354247}.
In addition, I was also involved in formalisation of single transferable vote, used in Australian Senate
\cite{10.1007/978-3-030-00419-4_4}.




\section{Current Work at Cambridge and Previous Work at Melbourne}
Currently, I am working as a Postdoctoral researcher at the university of Cambridge. 
In my current project \emph{Combinators for Algebraic Structures}\footnote{\url{https://www.cl.cam.ac.uk/~tgg22/CAS/}}, 
I am formalising various graph algorithms on \emph{semiring} algebraic 
structure and combinators (functions) to 
combine two, or more, algebraic structures. In this work, I am developing 
a mathematical correct-by-construction \cite{10.1007/978-3-319-66107-0_26} 
framework, in Coq thereom prover, based on theory of routing algebra 
\cite{10.1093/imamat/15.2.161, 10.1145/1080091.1080094} to alleviate network-engineers from 
proving the correctness of their protocol and focus entirely on protocol design.
All they need to do is express their protocol in our
framework, and it will tell what properties the protocol follows and what it does not. 
In addition, this framework can also be used in operation
research, given that its underlying principles are very similar to protocol design.



At the university of Melbourne, I worked on 
security concurrent separation logic for formally reasoning the information flow in concurrent programs. 
I used SecureC, tool developed at the university of Melbourne, to formalise an email server, 
an auction server, and a location server. All these works were proven to leaks no sensitive 
information to attackers, assuming that compiler respects all assumption (this work is under submission). 
In addition, I also developed a information flow secure gradient descent algorithm in SecureC for 
trusted execution environment, e.g., Intel SGX and ARM TrustZone. This work has been 
informally presented at PaveTrust\footnote{\url{https://www.acsac.org/2021/workshops/pavetrust/PAVeTrust-Program.pdf}}.


\section{Future Work}
My long-term aim is to make formal verification accessible and ubiquitous in 
software development, specifically for the software programs deployed in public domain. 
My expertise in \textbf{Theorem Proving, Election Security, and Cryptography}
gives me an unique perspective to solve challenging problem, that matters to many democracies. 

Throughout my past and current research, I have extensively used formal verification techniques 
to verify the software programs, and I would like to continue to do so in the future as well. 
In future, I would like to:

\begin{itemize}

\item focus on developing formally verified (electronic) voting software (components) programs 
for low coercion situations to promote direct democracy in Coq theorem prover. Switzerland often 
conduct referendums on various issues that matter to their democracy and encourages its citizens 
to participate actively in decision making. It would be interesting to design and 
implement electronic voting components, verified in Coq theorem prover, for low-coercion 
situations for various vote counting methods, e.g., Single Transferable Vote, First Past the Post, 
Instant-runoff, etc., on encrypted ballots. All these voting methods 
differs from each other so counting on encrypted ballot would be challenging task, while 
ensuring correctness, privacy, and verifiability.  
The rationale is that once we have a formally verified components, anyone --government or 
members of general public-- can use them for conducting elections, referendums, 
verifying election outcomes, etc. It would  encourage the voters to 
actively participate in various decision making process without any fear of vote selling.

\item focus on formally verified cryptographic primitives used in electronic voting, e.g., 
	sigma protocols (zero-knowledge-proof), verifiable (shuffling) mix-nets, etc., in Coq theorem prover. 
	In my all projects, I have formalised various 
	cryptographic primitives but ended up extracting OCaml code\footnote{We can extract OCaml/Haskell/Scheme 
	code from Coq formalisation.} and used OCaml compiler to 
	compile the code to machine level. However, OCaml compiler is not proven correct 
	and therefore it may introduce some bugs in our code. The challenging part of this project would be 
	to compile the formalised Coq cryptographic code to assembly, or binary, code all the way 
	down to machine level. It is highly non-trivial, but it opens the door of 
	collaboration with other research group working in verified compilation.

\item focus on formally verified decentralised peer-to-peer technical solution, inspired by 
\cite{liu2004linkable, noether2015ring}, in Coq theorem prover which will help 
whistleblowers in leaking documents and exposing 
corruption without revealing their identity. 
Being vocal against the government is one the most fundamental right of any citizen, but many 
authoritative governments do not appreciate dissent of any form. Therefore, it uses 
its powerful machinery to punish the dissident, in the name of national security. For example,
David McBridge, a former Australian Defence Force lawyer,  
is facing a threat, if charged guilty, of lifetime jail after
leaking the material alleging war crimes by members of the Australia's Special Operations
Task Group in Afghanistan (Australia is ranked very high in 
democracy index\footnote{\url{https://worldpopulationreview.com/country-rankings/democracy-countries}}). 
This reasearch opens the door of collaboration with many groups working in verified 
networking, verified cryptography, verified distributed systems, etc. 



\item focus on formally verified (computational) social choice theory in Coq theorem prover. 
	Voting methods admit many properties established by political scientists, social choice theorists, 
	and economists. For example, Schulze method follows Condorcet criterion, Reversal symmetry,
	Polynomial runtime, etc., so when we formalise Schulze method, or in fact any vote-counting method,
	in Coq theorem prover, we can push the boundary of correctness by proving that our 
	implementation of Schulze method also follows all the properties, i.e.,  Condorcet criterion, Reversal symmetry,
	Polynomial runtime, etc  \cite{tiwari2021machine}. 
	In addition, we can also analyse these 
	methods from computational point of view, also by formalising it in a Coq theorem prover, 
	that how easy from computational perspective to change the outcome of an election, etc. 
	This research opens the door of collaboration with political scientists, 
	social choice theorists, economists, game theorists, etc. 

	

\item focus on formally verified combinators for algebraic structure (CAS) in Coq theorem prover 
	(continue my collaboration with Timothy Griffin). 
	Currently, CAS formalisation is highly focused on networking protocols, but it can be 
	adapted for other areas, e.g., optimisation, clustering, algebraic program 
	analysis , etc. In the CAS, we use an abstract algebraic structure 
	\emph{semiring}\footnote{A set $R$ equipped with two binary operators $+$ and $*$, 
	known as (abstract) addition and (abstract) multiplication, $0$, (abstract) additive identity, 
	$1$, (abstract) multiplicative identity such that 
	($R$, $+$, $0$) is commutative monoid and ($R$, $*$, $1$) is monoid. 
	In addition, (abstract) multiplicative distribute over (abstract)  addition and 
	$0$ is (abstract) multiplicative annihilator.} as an underlying structure
	for computation and and we model various algorithm at the top this 
	structure. In this setting, an algorithm can compute different values depending 
	on the concrete structure of semiring. For example, the same algorithm 
	can compute shortest path, longest paths, 
	data flow of imperative programs \cite{gondran2008graphs, 10.1145/2500365.2500613}.
	

\end{itemize}



\bibliography{research-statement-bibliography} 
\bibliographystyle{unsrt}

\end{document}
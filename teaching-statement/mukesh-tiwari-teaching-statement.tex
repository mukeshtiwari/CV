\documentclass[a4paper]{article}
\title{Teaching Statement}
\author{Mukesh Tiwari}
%\date{\today}
\usepackage{url}
\setlength{\topmargin}{-10mm}
\setlength{\textwidth}{7in}
\setlength{\oddsidemargin}{-8mm}
\setlength{\textheight}{9in}
\setlength{\footskip}{1in}

\begin{document}
\fontsize{12}{15}
\selectfont
\maketitle


My teaching philosophy is to not immediately reach a solution but to develop a 
thinking process (problem solving mindset) that leads to the solution. I believe every 
student is different and has a unique style of learning, and my role is to help them find 
and hone their style.

\section{Teaching Experience}
When I started teaching as an assistant professor at the International Institute of Information 
Technology (IIIT), Bhubanesware, India\footnote{It is a small technical school, 
specilised for information technology.},
my single biggest challenge was keeping the students engaged in my class, especially the first year 
students in C programming course. At the IIIT, I taught C programming to 
first year students, Compiler Design and Java programming to third year students, and
Cryptography to final year (4th year) students. In each course, every single 
problem,  more or less, boiled down to keeping the students engaged in a topic. 
In order to keep them engaged in a class, I took a Coursera course on 
learning\footnote{\url{https://www.coursera.org/learn/learning-how-to-learn}} and 
read many academic articles and non-academic articles about effective learning.
I tried these techniques in my classes, and below I 
describe my experiences with teaching and efforts to engage my students. 





\subsection{Setting Clear Goals}
In every course, I started with the end goal of the course. For example, 
in C programming course, I told my students that by the end of this 
course they should be 
able write simple C programs, e.g., calculator, time tracking system,
student attendance system that automatically calculates their attendance percentage, 
validating a debit card, and other simple real-world problems. The rationale was 
to show a vision to excite them 
for learning and instill the feeling of empowerment, by a narrative from
being consumer to being creator of software programs.
In addition, in the beginning of a every single class 
I would tell my students explicitly the topics we were going to study and 
their importance. For example, when I taught pointers, I explained a 
very high level idea of an operating system and need to access memory 
in the course of booting the operating system. This demonstrated the 
need and importance of pointers. In addition, I told them
Linux is written in C and encouraged them to check the source code.
This two step process helped me on focusing the big picture, 
while at the same time tracking the progress towards the big picture.



\subsection{Start with Why}
One thing that I learnt by teaching for 3 years is that if you want a concept 
to stick in someone's mind, start with a \textit{why}\footnote{I learnt this idea by reading the book 
\textit{Start With Why} by Simon Sinek.}. For example, when I introduced functions 
in C programming course, I wanted to convey the idea of \textit{why do we need functions?} Therefore, I started the class by 
asking them to write a code, using pen and paper, to compute the \textbf{factorial of a number $n$}. 
Every one was comfortable with loops, so it was quick. I then asked them to write 
another program that computes the \textbf{$k^{th}$ power of the factorial of $n$}. 
In this assignment, some added an outer loop to cover the factorial computation, 
and some added another loop after the factorial computation (stored the factorial 
computation result and used it in later computation). 
At this point, I asked the class 
how one could write a program that computes the \textbf{factorial of 
the $k^{th}$ power of the factorial of $n$}. It slowly 
sinked to the class that they need to duplicate a lot of code, and once 
they had this revelation I introduced functions 
and showed them the solution of the previous problem by means of function composition.

\subsection{Catching my Mistakes}
To promote active participation, in the beginning of every class 
I would announce that there would be 
a few instances where I would write the solution of a problem wrong intentionally, 
and your goal is to catch me on the spot. If you succeed, you would be awarded 
class participation marks. If no one caught me, 
then I would tell the mistake, but no one would get any 
class participation marks. In every single class, especially in programming classes, 
I would first teach a topic and then solve problems on 
a blackboard related to the topic. 
It was during the problem solving where I made deliberate mistakes.
It increased the class participation by order of 
magnitude\footnote{I learnt this 
from advertising agencies where they write some ads wrong 
intentionally to grab the attention.}. In every single 
feedback that I got during my 3 years of teaching, 
almost everyone appreciated this idea of making deliberate mistakes. 



\subsection{Projects for Learning}
For every course that I taught, I designed projects related to the concepts 
and ideas present in the course. It was a part of my teaching philosophy to impart 
critical thinking. Furthermore, I asked my students come up with their own ideas to 
promote idea-exploration, a key step to develop a critical thinking and problem solving mindset. 
In the process of idea-exploration, students came up with various interesting 
ideas, but one that still stands out is a group of third year students that wanted 
to understand the workings of a a DNS server by writing their own. Even though they 
could not write a DNS server from scratch, they managed to understand 
most of the workings of a DNS server by downloading a C code from the Internet. 
I consider it as a trophy for myself. In addition, because of my own 
competitive programming background, 
I redesigned the C programming Lab and introduced them to competitive 
programming\footnote{\url{https://www.topcoder.com/community/arena}, 
\url{https://www.spoj.com/}}, which led to first ACM-ICPC team participation 
from IIIT. More importantly, it made many of the students curious about 
programming, which was considered a difficult subject.

\section{Potential Courses}
I feel qualified to teach most of the computer science courses 
because of my background in computer science, but 
my natural preferences are courses close to my research 
area, e.g., theorem proving, cryptography, logic 
and its applications in computer science, discrete 
mathematics, algorithms and data structures, 
introductory programming course (C/C++/Java/Python/Haskell/OCaml),
foundation of computing, etc. 
In addition, I would also like to design a course to teach 
various voting method used around the world and 
their pros and cons from a social choice theory perspective. 
Apart from these topics, I would 
be more than happy to teach other courses, given enough 
preparation time, at introductory and 
intermediate levels, e.g., type theory, theory of programming 
languages, networking, operating systems, databases, etc.



\end{document}
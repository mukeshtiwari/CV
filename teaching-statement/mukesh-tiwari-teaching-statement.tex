\documentclass[a4paper]{article}
\title{Teaching Statement}
\author{Mukesh Tiwari}
%\date{\today}
\usepackage{url}
\setlength{\topmargin}{-10mm}
\setlength{\textwidth}{7in}
\setlength{\oddsidemargin}{-8mm}
\setlength{\textheight}{9in}
\setlength{\footskip}{1in}

\begin{document}
\fontsize{12}{15}
\selectfont
\maketitle


My teaching philosophy is not to immediately reach a solution but to develop a 
thinking process (problem solving mindset) that leads to the solution. I believe every 
student is different and has a unique style of learning, and my role is to help them find 
and hone their style.

When I started teaching as an 
assistant professor in International Institute of Information 
Technology (IIIT), Bhubanesware, India\footnote{It is a small technical school, specilised for information technology.},
I faced the biggest obstacle of keeping the students engaged, specially the first year 
students in C programming course. At the IIIT, I taught C programming to 
first year students, Compiler Design and Java programming to third year students, and
Cryptography to final year (4th year) students. In each course, more or less, every single 
problem boils down to keeping the students engaged in the topic. 
In order to keep them engaged in a class, I ready many articles, 
took a Coursera course on learning\footnote{\url{https://www.coursera.org/learn/learning-how-to-learn}}, and 
reading academic papers (cite one) about effective learning. Below are few of 
my experiences that I would like continue as teacher.



\section{Setting Clear Goals}
In every course, I started with the end goal of the course. For example, 
in C programming course, I told them by end of this course you should be 
able write simple C programs, e.g., calculator, time tracking system for IIIT employees,
student attendance system that automatically calculates if their attendance percentage, 
etc. In addition, in every single class I would tell them explicitly what 
were we going to study and why it matters. 



applications (give here some concrete). In addition, 
in every individual class I also mentioned that by end of this 
class, you should be comfortable with this idea. 

\section{Start with Why}
One thing that I learnt by teaching for 3 years is that if you want a concept 
to stick in someone's mind, start with a \textit{why}\footnote{I learnt this idea by reading the book 
\textit{Start With Why} by Simon Sinek.}. For example, when I introduced functions 
in C programming course, I wanted to convey the idea of \textit{why do we need functions?}. I started the class by 
asking them to write a code, using pen and paper, to compute the factorial of a number $n$. 
Every one was comfortable with loop, so it was quick. Then I asked them to write a program that 
computes the $k^{th}$ power of the factorial of $n$. Everyone reused the code from 
the previous assignment and some added one more loop at the top of the factorial 
computation, and others wrote a new loop after the factorial computation. 
At point, I asked them how would you write a program if I asked to write 
a function that computes the factorial of the $k^{th}$ power of the factorial of a number. 
What about the $k^{th}$ power of the factorial of the $k^{th}$ power of the factorial of a number? 
At this point, everyone get the problem, code repetition. Finally, I taught them functions 
and showed them the solution of the previous problem by means of function composition. 

\section{Catching my Mistakes}
In order to promote active participation, I told the class, that I was teaching, in every single lecture that there would be 
few instances where I would write the solution of a problem wrong intentionally, and you had to 
caught me on spot. If you succeed, you would be awarded a class participation marks. If no one could catch me, 
then I would tell what mistake I made but no one would get any class participation marks. It increased the class 
participation by order of magnitude\footnote{I learnt this from advertising agency where they write some ads wrong 
intentionally to grab the attention.}. 




\section{Projects for Learning}
For every single course that I taught, I designed projects related to the concepts 
and ideas present in the course. It was a part of my teaching philosophy to impart 
critical thinking. Furthermore, I asked them come up with their own ideas to 
promote idea-exploration, a key step to develop a critical thinking and problem solving mindset. 
In the process of idea-exploration, students came up with various interesting ideas, but 
one that still stands out is a group of first year students wanted to understand the workings of DNS server 
by writing their own. Even though, they could not write their own because of lack knowledge of many
other concepts needed to write DNS server, but they managed to understand the most of the 
DNS server workings by downloading a C code from the Internet. I consider it as a trophy 
for myself. Because of my competitive programming background, 
I redesigned the C programming Lab and introduced them to competitive 
programming\footnote{\url{https://www.topcoder.com/community/arena}, \url{https://www.spoj.com/}}, which led to first 
ACM-ICPC team participation from IIIT. Moreover, it started a culture 
of programming, which ultimately led to participation in \textit{Google Summer of Code}. 



\section{Potential Courses}
I feel qualified to teach most of the computer science course because of my background in computer science, but 
my natural preferences are courses close to my research area, e.g., theorem proving, cryptography, logic 
and its applications in computer science, discrete mathematics, algorithms and data structures, 
introductory programming course (C/C++/Java/Python/Haskell/OCaml).
In addition, I would also like to design a course to teach various voting method used around the world and 
their pros and cons from social choice theory perspective. Apart from these topics, I would 
be more than happy to teach other courses, given enough preparation time, at introductory and 
intermediate levels, e.g., type theory, theory of programming languages, 
networking, operating systems, databases, etc.



\section{Diversity}
I was born in a remote Indian village in a relatively wealthy family 
because unlike many of friend I never needed to go work in
agricultural farms to ensure ends meet. My father was 
a government school teacher in a small town \textit {Mughalsarai} with many nice 
schools. After studying in my village for 4 years, I moved 
to Mughalsarai to continue my studies. I was good in studies, or at least 
that is what I thought for a very long time until I moved to a prestigious university for 
my undergraduate studies. In my undergraduate studies, I met more students who 
were more privileged than me. It dawn upon me that the only reason 
I was at that university because I was lucky to born in family which 
has a lot of resources, unlike to many of village-friends. Most of 
them were good but they never got enough time to study, and 
slowly and slowly they lost the interest.




and after spending my first 8 years 
in my village, I moved to small town for education. 
I was good 
at studying and I thought I was talented, at least for a very long time I thought like it.
After a long time, I realised that I was just lucky because I had a lot of time 
for studying, unlike my friends. Diversity matters a lot to me 
because many of my friends who studied with me were also but they 
could not pursue any further education but they could have 
if they were given a chance. 

\begin{itemize}
  \item Having a clear Goal in the beginning of semester, in addition 
   to goals in every class
  \item Using the trick of advertising agencies
  \item Giving them a sense of responsibility if they don't the job properly, 
    some bugs may be costly 
  \item Encouraging them participate by challenging them with a right problem 
  \item 
\end{itemize}

 
\cite{10.1007/978-3-030-00419-4_4}
\bibliographystyle{plain}
\bibliography{teaching-statement.bib}
\end{document}
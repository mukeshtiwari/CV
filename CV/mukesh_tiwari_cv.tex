%% start of file `template.tex'.
%% Copyright 2006-2015 Xavier Danaux (xdanaux@gmail.com).
%
% This work may be distributed and/or modified under the
% conditions of the LaTeX Project Public License version 1.3c,
% available at http://www.latex-project.org/lppl/.


\documentclass[11pt,a4paper,sans]{moderncv}        % possible options include font size ('10pt', '11pt' and '12pt'), paper size ('a4paper', 'letterpaper', 'a5paper', 'legalpaper', 'executivepaper' and 'landscape') and font family ('sans' and 'roman')

   
% moderncv themes
\moderncvstyle{casual}                             % style options are 'casual' (default), 'classic', 'banking', 'oldstyle' and 'fancy'
\moderncvcolor{blue}                               % color options 'black', 'blue' (default), 'burgundy', 'green', 'grey', 'orange', 'purple' and 'red'
%\renewcommand{\familydefault}{\sfdefault}         % to set the default font; use '\sfdefault' for the default sans serif font, '\rmdefault' for the default roman one, or any tex font name
%\nopagenumbers{}                                  % uncomment to suppress automatic page numbering for CVs longer than one page
% character encoding
\usepackage[utf8]{inputenc}                       % if you are not using xelatex ou lualatex, replace by the encoding you are using
%\usepackage{CJKutf8}                              % if you need to use CJK to typeset your resume in Chinese, Japanese or Korean

% adjust the page margins
\usepackage[scale=0.75]{geometry}
%\usepackage[backend=biber]{biblatex}
%\setlength{\hintscolumnwidth}{3cm}                % if you want to change the width of the column with the dates
%\setlength{\makecvtitlenamewidth}{10cm}           % for the 'classic' style, if you want to force the width allocated to your name and avoid line breaks. be careful though, the length is normally calculated to avoid any overlap with your personal info; use this at your own typographical risks...






% personal data
\name{Mukesh}{Tiwari}
                              % optional, remove / comment the line if not wanted
%\address{street and number}{postcode city}{country}% optional, remove / comment the line if not wanted; the "postcode city" and "country" arguments can be omitted or provided empty
\phone[mobile]{+44-7824648138}                   % optional, remove / comment the line if not wanted; the optional "type" of the phone can be "mobile" (default), "fixed" or "fax"
%\phone[fixed]{+2~(345)~678~901}
%\phone[fax]{+3~(456)~789~012}
\email{mukesh.tiwari@swansea.ac.uk, mukeshtiwari.iiitm@gmail.com}                               % optional, remove / comment the line if not wanted
\homepage{mukeshtiwari.github.io/}                         % optional, remove / comment the line if not wanted
\social[linkedin]{mukesh-tiwari-3609486/}                        % optional, remove / comment the line if not wanted
%\social[twitter]{mukesh\_tiwari}                             % optional, remove / comment the line if not wanted
\social[github]{mukeshtiwari}                              % optional, remove / comment the line if not wanted
%\extrainfo{additional information}                 % optional, remove / comment the line if not wanted
%\photo[64pt][0.4pt]{DP.jpg}                       % optional, remove / comment the line if not wanted; '64pt' is the height the picture must be resized to, 0.4pt is the thickness of the frame around it (put it to 0pt for no frame) and 'picture' is the name of the picture file
                                % optional, remove / comment the line if not wanted

% bibliography adjustements (only useful if you make citations in your resume, or print a list of publications using BibTeX)
%   to show numerical labels in the bibliography (default is to show no labels)
\makeatletter\renewcommand*{\bibliographyitemlabel}{\@biblabel{\arabic{enumiv}}}\makeatother
%   to redefine the bibliography heading string ("Publications")
%\renewcommand{\refname}{Articles}




% bibliography with mutiple entries
%\usepackage{multibib}
%\newcites{book,misc}{{Books},{Others}}
%----------------------------------------------------------------------------------
%            content
%----------------------------------------------------------------------------------
\begin{document}
%\begin{CJK*}{UTF8}{gbsn}                          % to typeset your resume in Chinese using CJK
%-----       resume       ---------------------------------------------------------
\makecvtitle

\section{Education}
\cventry{2016--2020}{PhD, Computer Science}{Australian National University}{Canberra}{Australia}{}  % arguments 3 to 6 can be left empty
\cventry{2004--2009}{Integrated Postgraduate}{Indian Institute of Information Technology \& Management}{Gwalior}{India}{}

\section{PhD thesis}
\cvitem{Title}{\emph{Formally Verified Verifiable Electronic Voting Scheme}}
\cvitem{Supervisor}{Dirk Pattinson}
\cvitem{Description}{We focussed on three main challenges posed by electronic voting: 
correctness, privacy, and verifiability. We addressed correctness by using 
a theorem prover to implement a vote-counting algorithm, 
privacy by using homomorphic encryption,  and verifiability 
by generating a independently checkable scrutiny sheet. 
Our work had been formalised in the Coq theorem prover. }
\section{Employment}
\cventry{2024-}{Lecture}{Swansea University}{Swansea}{United Kingdom}
  {I am exploring the use of zero-knowledge succinct noninteractive arguments of knowledge (ZKSNARK) in electronic voting and 
  anonymous credentials. The goal is to replicate the confidentiality and integrity of a 
  traditional paper-ballot election in an electronic format.}
\cventry{2023-2024}{Senior Research Associate}{University of Oxford}{Oxford}{United Kingdom}
  {I worked on connecting symbolic models of cryptographic protocols and their verified implementations using 
session types. The goal was to obtain a mathematically proven correct distributed implementation that 
mimics the real-world situation.}
\cventry{2021-2023}{Senior Research Associate}{University of Cambridge}{Cambridge}{United Kingdom}
  {I worked on formalising a networking-protocol framework based on semiring algebraic structure in 
  the Coq theorem prover. The goal was to develop a mathematical proven correct framework so that a protocol 
  designer could assess the properties of their protocols using my framework.}
\cventry{2020-21}{Research Associate}{University of Melbourne}{Melbourne}{Australia}
  {I worked with Toby Murray  on \textit{Security Concurrent Separation Logic}. 
  The aim was 
  to mathematically reason about memory safety and information flow property of concurrent programs written in C.}
\cventry{2018-20}{Tutor}{Australian National Univeristy}{Canberra}{Australia}
  {I was a tutor for a first year logic course and Haskell programming course. My role was to 
  help students understand the concepts, clearing their doubts, and assisting them in homework.}
\cventry{2013--2015}{Lecturer}{International Institute of Information Technology}{Bhubaneswar}{India}
{This role was primarily teaching focussed, and 
the courses I taught were \textit{C programming, Java Programming, Web Technologies, 
Compiler Design,} and \textit{Cryptography}.  
In addition,  every year I supervised two master's students in their final year project.}

%\subsection{Miscellaneous}
\cventry{2012--2013}{Haskell Developer}{Parallel Scientific}{Colorado}{USA}
{In this role, my primary job was  research and prototype high performance software programs, mainly linear algebra algorithms
written in Haskell.}

\cventry{2009--2012}{Technical Assistant}{Government of India}{Kolkata}{India}
{I worked as a developer for automating the day-to-day job,  including enforcing the security policies of the organisation.}

\cventry{2008--2008}{Summer Intern} {Arcelor-Mittal, Research \& Development Technological Centre} {Avil{\'e}s} {Spain} 
{I worked on formalising many business requirements into a linear programming problem and wrote a 
 custom interface that interacted with Arcelor-Mittal's in-house linear programming solver.}

%\section{Languages}
%\cvitemwithcomment{Language 1}{Skill level}{Comment}
%\cvitemwithcomment{Language 2}{Skill level}{Comment}
%\cvitemwithcomment{Language 3}{Skill level}{Comment}


\section{Conference Publication}
\cvitem{[1]}{{Toby Murray, Mukesh Tiwari, Gidon Ernst, and David A. Naumann. 
    Assume but Verify: Deductive Verification of Leaked Information in Concurrent Applications. 
    In Proceedings of the 2023 ACM SIGSAC Conference on 
    Computer and Communications Security, CCS '23, Copenhagen, Denmark, 26-30 Nov.}
    {\url{https://github.com/mukeshtiwari/IFMachine/}}.
    {(co-developer with Toby Murray and Gidon Ernst. In this work, I formally verified the case studies: 
    differentially private location-server, federated machine learning 
    server, aution-server, and email-server in SecureC; project duration: 1.6 years).}}

\cvitem{[2]}{{Thomas Haines, Rajeev Gor{\'e}, and Mukesh Tiwari.
    Machine-checking Multi-Round Proofs of Shuffle:Terelius-Wikstrom and Bayer-Groth. 
    32nd USENIX Security Symposium (USENIX 2023), Anaheim, California, USA, August 9-11, 2023.}
    {\url{https://github.com/mukeshtiwari/secure-e-voting-with-coq}}.
    (co-developer with Thomas Haines. I proved facts related to zero-knowledge proof in 
    the Coq theorem prover; project duration: 2 years).}


\cvitem{[3]}{{Nadim Kobeissi, Georgio Nicolas, and Mukesh Tiwari. 
    Verifpal: Cryptographic Protocol Analysis for the Real World. In Karthikeyan 
    Bhargavan, Elisabeth Oswald, and Manoj Prabhakaran, editors, Progress in Cryptology - INDOCRYPT 2020, 
    pages 151--202, Cham, 2020. Springer International Publishing.}
    (co-developer with Georgio Nicolas. I worked on proofs related to Verifpal model in Coq;
    project duration: 8 months)}
    
\cvitem{[4]}{{Thomas Haines, Rajeev Gor{\'e}, and Mukesh Tiwari. 
    Verified Verifiers for Verifying Elections. In Proceedings of the 2019 ACM SIGSAC Conference on 
    Computer and Communications Security, CCS '19, page 685--702, New York, NY, 
    USA, 2019. Association for Computing Machinery.}
    {\url{https://github.com/mukeshtiwari/secure-e-voting-with-coq}}.
    (co-developer with Thomas Haines. I worked on efficient finite field arithmetic, required 
    for efficient zero-knowledge proof validation of well-formedness of a ballot;
    project duration: 1 year)}

\cvitem{[5]}{{Thomas Haines, Dirk Pattinson, and Mukesh Tiwari.} 
    {Verifiable Homomorphic Tallying for the Schulze Vote Counting Scheme, 
    In Verified Software: Theories, Tools, and Experiments. Springer, 2019.}
    {\url{https://github.com/mukeshtiwari/EncryptionSchulze/tree/master/code/Workingcode}}
    (lead developer, project duration: 2 years)}


\cvitem{[6]}{{Milad K. Ghale, Rajeev Gor{\'e}, Dirk Pattinson, and Mukesh Tiwari.
    Modular Formalisation and Verification of STV Algorithms. In Robert Krimmer, Melanie Volkamer, 
    V{\'e}ronique Cortier, 
    Rajeev Gor{\'e}, Manik Hapsara, Uwe Serd{\"u}ltt, and David Duenas-Cid, editors, 
    Electronic Voting, pages 51--66, Cham, 2018. Springer International Publishing.}
    {\url{https://github.com/mukeshtiwari/Modular-STVCalculi}}.
    (co-developer with Milad K. Ghale. I proved some of the critical theorems, required for 
    code extraction; project duration: 8 months)}

\cvitem{[7]}{{Lyria Bennett Moses, Rajeev Gor{\'e}, Ron Levy, Dirk Pattinson, and Mukesh Tiwari.}
    {No More Excuses: Automated Synthesis of Practical and Verifiable Vote-Counting Programs for Complex Voting Schemes.}
    {In Robert Krimmer, Melanie Volkamer, Nadja Braun Binder, Norbert Kersting, Olivier Pereira, and Carsten Sch{\"u}rmann, 
    editors, Electronic Voting, pages 66--83, Cham, 2017. Springer International Publishing.}
    {\url{https://github.com/mukeshtiwari/formalized-voting/tree/master/SchulzeOCaml}}
    (lead developer, project duration: 8 months)}

\cvitem{[8]}{{Dirk Pattinson and Mukesh Tiwari.} {Schulze Voting as Evidence Carrying Computation}. 
    {In Mauricio Ayala-Rinc{\'o}n 
    and C{\'e}sar A, Mu{\~{n}}oz, editors, \textit{Interactive Theorem Proving}, pages 410--426.
    Cham, 2017. Springer International Publishing.}
    {\url{https://github.com/mukeshtiwari/formalized-voting/blob/master/paper-code}} 
    (lead developer, project duration: 1 year)}



\cvitem{[9]}{{Mukesh Tiwari, Karm V. Arya, Rahul Choudhari, and Kumar S. Choudhary. 
    Designing Intrusion Detection to Detect Black Hole and Selective Forwarding Attack in 
    WSN Based on Local Information. In 2009 Fourth International Conference on Computer 
    Sciences and Convergence Information Technology, pages 824--828, Nov 2009.}
    (lead developer; project duration: 1 year)}

\cvitem{[10]}{{Rahul Choudhari, Karm V. Arya, Mukesh Tiwari, and Kumar S. Choudhary. 
    Performance Evaluation of SCTP-Sec: A Secure SCTP Mechanism. In 2009 Fourth 
    International Conference on Computer Sciences and Convergence Information Technology, 
    pages 1111--1116, Nov 2009.}
    (co-developer with Rahul Choudhari; project duration: 1 year)}
    

%\nocite{*}
%\bibliographystyle{plainyr-rev}
%\bibliography{publications}

\section{Workshop Publications}
\cvitem{[1]}{{Mukesh Tiwari and Dirk Pattinson}.
        {Machine Checked Properties of the Schulze Method}.
        {7th Workshop on Hot Issues in Security Principles and Trust} {2021}.}
\cvitem{[2]}{{Mukesh Tiwari}. {Towards Leakage-Resistant Machine Learning in Trusted Execution Environments}.
    {Program Analysis and Verification on Trusted Platforms (PAVeTrust) Workshop} {2021}.}
\cvitem{[3]}{{Nadim Kobeissi, Georgio Nicolas, and Mukesh Tiwari.}
    {Verifpal: Cryptographic Protocol Analysis for the Real World.}
    {Proceedings of the 2020 ACM SIGSAC Conference on Cloud Computing Security Workshop} {2020}.}
\cvitem{[4]}{{Gidon Ernst, Toby Murray, and Mukesh Tiwari}.
    {Verifying the Security of a PGP Keyserver}. {VerifyThis challenge 2020}.}

\section{Work in Progress}
\cvitem{[1]}{Modelling Networking Protocols Mathematically. In this work, we develop 
    a formally verified framework that researchers can use to verify 
    the properties of their protocols (joint work Nobuko Yoshida and Mina Cyrus. In this work, 
    I formally verified generalised graph algorithm on semiring  algebra in 
    the Coq theorem prover). (submitted to TACAS 2025).
    \url{https://github.com/mukeshtiwari/Semiring_graph_algorithm}}   

\cvitem{[2]}{Formally Verified Verifiable Group Generators. In this work, we develop a formally 
    verified algorithm that can be used to bootstrap a democratic election (joined work with Mina Cyrus). (submitted to FSEN 2025).
    \url{https://github.com/mukeshtiwari/Formally_Verified_Verifiable_Group_Generator}.}

       
\cvitem{[3]}{An Algebraic Framework for Multi-Objective Optimisation. In this work, we develop 
    a formally verified framework in the Coq theorem prover that can be used to model 
    various multi-objective 
    optimisation problem as a graph algorithm in the semiring framwork. 
    (sole author). (ongoing).
    \url{https://github.com/mukeshtiwari/Formally-Verified-MultiObjective-Optimisation}}   
       
\cvitem{[4]}{Theorem Provers to Protect Democracies. In this work, we are formalising 
    all the cryptographic components written in Java of SwissPost in the Coq theorem
    prover. Our goal is to replace the SwissPost Java 
    implementations ({\url{https://bit.ly/3EODmnF}})
	with mathematically proven correct Coq implementations to
    write an independent verifier for the scrutiny sheet of elections conducted by 
    Swiss Post software programs (joint work with Bas Spitters and 
    Berry Schoenmakers) (work in progress and planning to submit to IEEE S\&P 2024).
    \url{https://github.com/mukeshtiwari/Dlog-zkp}.}
\cvitem{[5]}{{Machine Checked Properties of the Schulze Method. In this work, 
    we are formally verifying all the (social choice) properties of the Schulze method.
    (lead developer, joint work with Dirk Pattionson) (work in progress).}
    {\url{https://github.com/mukeshtiwari/Schulzeproperties}}.}


\section{Community Service (Reviewer)}
\cvitem{[1]}{ACM Transactions on Privacy and Security}
\cvitem{[2]}{Sadhana - Academy Proceedings in Engineering Sciences}
\cvitem{[3]}{IEEE Security \& Privacy}
    
\section{Skills}
\cvitem{Coding}{Coq, Lean, Haskell, OCaml, Python, C, Racket}
\cvitem{Language}{Hindi, English}

%\section{Interests}
%\cvitem{hobby 1}{Description}
%\cvitem{hobby 2}{Description}
%\cvitem{hobby 3}{Description}

%\section{Extra 1}
%\cvlistitem{Item 1}
%\cvlistitem{Item 2}
%\cvlistitem{Item 3. This item is particularly long and therefore normally spans over several lines. Did you notice the indentation when the line wraps?}

%\section{Extra 2}
%\cvlistdoubleitem{Item 1}{Item 4}
%\cvlistdoubleitem{Item 2}{Item 5\cite{book1}}
%\cvlistdoubleitem{Item 3}{Item 6. Like item 3 in the single column list before, this item is particularly long to wrap over several lines.}

\section{Awards}
\cvitem{} {HDR Fee Remission Merit Scholarship}
\cvitem{} {ANU PhD Scholarship (International)}
\cvitem{} {Full Scholarship to attend DeepSpec Summer School 2018, Princeton University}
\cvitem{} {Travel Scholarship to attend Marktoberdorf Summer School 2019}

\section{References}
\begin{cvcolumns}
  \cvcolumn{}{\begin{itemize}\item Dirk Pattinson, Research School of Computer Science, Australian National University, Canberra, dirk.pattinson@anu.edu.au
   \end{itemize}}
   \end{cvcolumns}
\begin{cvcolumns}
    \cvcolumn{}{\begin{itemize}\item Toby Murray, School of Computing and Information Systems, University of Melbourne, Melbourne, toby.murray@unimelb.edu.au
     \end{itemize}}
\end{cvcolumns}
\begin{cvcolumns}
    \cvcolumn{}{\begin{itemize}\item Nobuko Yoshida, Department of Computer Science, University of Oxford, Oxford, nobuko.yoshida@cs.ox.ac.uk
     \end{itemize}}
\end{cvcolumns}
\begin{cvcolumns}
    \cvcolumn{}{\begin{itemize}\item Thomas Haines, Research School of Computer Science, Australian National University, Canberra, thomas.haines@anu.edu.au
     \end{itemize}}
  \end{cvcolumns}



%\section{Publications}


% Publications from a BibTeX file without multibib
%  for numerical labels: \renewcommand{\bibliographyitemlabel}{\@biblabel{\arabic{enumiv}}}% CONSIDER MERGING WITH PREAMBLE PART
%  to redefine the heading string ("Publications"): \renewcommand{\refname}{Articles}

    
%\section{Publications}
%\subsection{Journals}
%\cvitem{2000}{P.~Student. My Paper...}                    % 'publications' is the name of a BibTeX file

% Publications from a BibTeX file using the multibib package
%\section{Publications}
%\nocitebook{book1,book2}
%\bibliographystylebook{plain}
%\bibliographybook{publications}                   % 'publications' is the name of a BibTeX file
%\nocitemisc{misc1,misc2,misc3}
%\bibliographystylemisc{plain}
%\bibliographymisc{publications}                   % 'publications' is the name of a BibTeX file

\clearpage
%-----       letter       ---------------------------------------------------------
% recipient data
%\recipient{Company Recruitment team}{Company, Inc.\\123 somestreet\\some city}
%\date{January 01, 1984}
%\opening{Dear Sir or Madam,}
%\closing{Yours faithfully,}
%\enclosure[Attached]{curriculum vit\ae{}}          % use an optional argument to use a string other than "Enclosure", or redefine \enclname
%\makelettertitle

%Lorem ipsum dolor sit amet, consectetur adipiscing elit. Duis ullamcorper neque sit amet lectus facilisis sed luctus nisl iaculis. Vivamus at neque arcu, sed tempor quam. Curabitur pharetra tincidunt tincidunt. Morbi volutpat feugiat mauris, quis tempor neque vehicula volutpat. Duis tristique justo vel massa fermentum accumsan. Mauris ante elit, feugiat vestibulum tempor eget, eleifend ac ipsum. Donec scelerisque lobortis ipsum eu vestibulum. Pellentesque vel massa at felis accumsan rhoncus.
%
%Suspendisse commodo, massa eu congue tincidunt, elit mauris pellentesque orci, cursus tempor odio nisl euismod augue. Aliquam adipiscing nibh ut odio sodales et pulvinar tortor laoreet. Mauris a accumsan ligula. Class aptent taciti sociosqu ad litora torquent per conubia nostra, per inceptos himenaeos. Suspendisse vulputate sem vehicula ipsum varius nec tempus dui dapibus. Phasellus et est urna, ut auctor erat. Sed tincidunt odio id odio aliquam mattis. Donec sapien nulla, feugiat eget adipiscing sit amet, lacinia ut dolor. Phasellus tincidunt, leo a fringilla consectetur, felis diam aliquam urna, vitae aliquet lectus orci nec velit. Vivamus dapibus varius blandit.
%
%Duis sit amet magna ante, at sodales diam. Aenean consectetur porta risus et sagittis. Ut interdum, enim varius pellentesque tincidunt, magna libero sodales tortor, ut fermentum nunc metus a ante. Vivamus odio leo, tincidunt eu luctus ut, sollicitudin sit amet metus. Nunc sed orci lectus. Ut sodales magna sed velit volutpat sit amet pulvinar diam venenatis.
%
%Albert Einstein discovered that $e=mc^2$ in 1905.
%
%\[ e=\lim_{n \to \infty} \left(1+\frac{1}{n}\right)^n \]

%\makeletterclosing

%\clearpage\end{CJK*}                              % if you are typesetting your resume in Chinese using CJK; the \clearpage is required for fancyhdr to work correctly with CJK, though it kills the page numbering by making \lastpage undefined
\end{document}


%% end of file `template.tex'.

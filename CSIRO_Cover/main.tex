%Author by Rajib Das Bhagat (rajibdasbhagat@gmail.com)
%
\documentclass[11pt,a4paper,roman]{moderncv}      
\usepackage[english]{babel}

\moderncvstyle{classic}                            
\moderncvcolor{black}                            

% character encoding
\usepackage[utf8]{inputenc}
\usepackage{url}

% adjust the page margins
\usepackage[scale=0.90]{geometry}

% personal data
\name{Mukesh Tiwari}{}
\email{mt883@cam.ac.uk}
\phone[mobile]{+447824648138}               
\address{Cambridge, United Kingdom}


\begin{document}

\recipient{To}{Dr. Surya Nepla \\
               Group Leader, \\CSIRO's Data61 \& Deputy Research Director Cybersecurity}
\date{\today}
\opening{\textbf{Application for the post of Research Scientist in Cybesecurity (74837)}}
\closing{Your Sincerly, \vspace{-1em}}



\makelettertitle



Dear Dr. Surya Nepal, 
\\
%references such as what and how you got this information
\vspace{1em}
I am writing to apply
the job \textbf{Research Scientist in Cybersecurity}. I have an extensive experience in
security research, and I find that CSIRO would be a perfect place to continue 
my research and expand my horizons in other areas, including machine 
learning, AI, etc.  I have a PhD from the Australian National University, Canberra
and have been working as a Senior Research Fellow at the University of 
Cambridge since October 2021. Before moving to Cambridge, I was a 
research fellow at the University of Melbourne. 

\vspace{0.5cm}
In my PhD, I have verified the Schulze vote counting method, a
 widest path problem. I have addressed ballot privacy by using
 homomorphic encryption, and verifiability by means of producing an
 independently verifiable scrutiny sheet, consisting of various zero-knowledge-proofs, 
 the validity of which can
 be independently substantiated, that witnesses the correctness of the
 execution of an election. At CSIRO, as a Research Scientist, 
 I would like to expand my research area  
 in to cybersecurity using machine learning and AI, in addition to carrying 
 out the cybersecurity using formal method.
 
 

\vspace{0.5cm}
As a senior research associate at the University of Cambridge, 
I am working on  
a mathematical correct-by-construction framework based on theory of routing algebra 
to alleviate network-engineers from proving the 
correctness of their protocol and focus entirely on protocol design.
All  they need to do is express their protocol in my (mathematical) 
framework, and it will 
tell what property the protocol follows and what it does not. 
In addition, my framework can also be used in operation
research, given that its underlying principles are very similar to protocol design.


As a research associate at the University of Melbourne, I did
 acquire hands-on knowledge of separation logic and information flow
 security. I have spearheaded three projects:
 (i) A formally verified auction server, (ii) A formally
 verified location server, and (iii) A formally verified machine learning 
 algorithm that is resistant to side-channel attacks and can  
 run inside the Intel SGX (Software Guard Extensions) enclave for learning
 on senstive data. All three implementations have been proven 
 memory safe (using separation logic) and free from information
 leaks (applying information flow security), using the SecCSL tool.

\vspace{0.5cm}
I look forward to hearing from you. Let me know if you have any questions. \\
 

\vspace{0.5cm}


\makeletterclosing

\end{document}

